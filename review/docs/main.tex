\section{state of the art of interacting with large displays}
We present an overview of research in interaction techniques for cross device interaction, with an emphasis on the natural user interface.\\

By looking at Public Space we found the following topics in HCI: Cross-device Interaction, Large Displays, Multi-device Interaction, Natural User Interface, Proxemics, Social Interaction, Tabletop Interaction, and Tangible Interaction. We used these topics for our inquiry in ACM DigitalLibrary, SpringerLink and Microsoft Research, which matched a total of 56 papers.\\
 
To narrow down our focus we selected papers that matched a set of strict criteria: (1) The paper should contain one or more interaction techniques, (2) It is important that the techniques are evaluated using either interviews, surveys, field studies or any other research method.\\
The final set contained 34 literature sources which we further examined, by writing a short summary for each paper which included "title, author's thesis (aim of paper), objectives (goals, summaries), methodologies used, findings, conclusion, and keywords". Using applied thematic analysis different themes were induced on the summaries, which we later categorized into 5 themes: interaction with large displays, interaction with public displays, natural user interactions, cross-device interactions, cross-device natural user interactions.\\

The identified themes and the specific papers correlated to them are presented in Table 1, with full citations in the references section of this paper. 

\subsection{Interacting with Large Displays}
Influenced by Weiser’s paper \cite{Weiser:1991} in which he states that ubiquitous computers have to be of different sizes, each suited to a particular task. As real power emerges from the interaction of all the different devices, researchers started pushing the boundaries towards constructing large displays used horizontally as tabletop displays or vertically as display-walls.\\
 
Welnner showed the first examples of a display used as a tabletop -  Digital Desk[32] he utilized optical and acoustic finger detection on the tabletop and also played with the possibility of tactile manipulation of physical objects on the surface. Portfolio Wall was developed at the end of the 90s as a system that utilized a vertical touch-screen display by Buxton[33]. His idea was proposed as a corkboard in digital format used to share work inside the design team.\\

Rapid progression in the technology for displays and projections, naturally led to lowering of the marginal price, thus "large displays have moved out of research laboratories into public spaces such as museums, libraries, plazas, and architectural facades, where they present information and enhance experiences" [3]\\

We can summarize that Public displays is an application of large display therefore examining the latter without the first is a futile action.

