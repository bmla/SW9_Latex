% !TEX root = ../paper.tex
\section{Introduction}
The evolution of ways people interact with the digital has been momentous and notable in the short life-span of technology. Not only has the perception of device use, and device choice, but also the context we use devices has been under transformation multiple times. From portable personal computers, originally considered mostly for specialized field applications, such as in the military, for accountancy, or for sales representatives, which addressed mobility of a person's workspace, to modern hand-held devices which presented, their users, such degree of freedom that ultimately workspaces are starting to deteriorate. This expansion has not only increased mobile computing, but also made it widespread. \\

As divergent technologies and devices are being adopted in different domains cross-device interaction is currently becoming more important and relevant, on account of people bring their hand-held devices with them everywhere. This means that going out is not only a person taking a walk, but also a hand-held device. This ubiquitous presence of devices means that it can be leverage to enhance ordinary situations. E.g. by providing a mediator, now when you go to the cinema early, you don't have to just sit you can play a game on the screen with the present people inside. So this emergence of new technology, we need to understand how different interaction techniques perform in use, e.g. in terms of usability, like task completion time, or in terms of user experience, like user satisfaction. \\

Research in the area of cross-device interaction has been trying to keep up with the changing trends, spawning it's root in the late 90's in ubiquitous computing with Rekimoto's papers, his work focused on data transfer between multiple devices, he designed a interaction technique where using a Pen it was possible to physically pick up a digital object from one screen, carry it through real space, and drop it in a different place - typically a different computer screen \cite{Rekimoto:1997}, in later work of his he argued for, what he called, multi computed user interface and that interaction technique must overcome the boundaries among devices in multi-device settings\cite{Rekimoto:1998}.

Even in the early work of Rekimoto we can see this natural component in his technique design, recent HCI research has focused on how to include it in the cross-device section, hence adding an extra natural modality contributing to what should be know as cross-device natural user interaction.  Some researchers used spatial information \cite{Marquardt:2011, Marquardt:2012}, others used touch \cite{Seifert:2012}, and third used combined touch with air gestures \cite{Bragdon:2011} . But we still have limited understanding of how to design cross-device natural user interaction techniques and also we are missing on quantitative studies.\\

Inspired by the great opportunities presented of such challenges, this paper reports on a quantitative study between four different cross-device natural user interaction techniques for data transfer between hand-held device and large displays. To our knowledge the only research close to ours is done by Bragton et. al. \cite{Bragdon:2011}, however distinguished difference are cleared seen in the settings - private vs public, moderation vs none on the large display, and qualitative vs quantitative study.
\todo{write what we find in findings and discussion when we write them}

