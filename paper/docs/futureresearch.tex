% !TEX root = ../paper.tex
\section{Future Research} \label{sec:futureresearch}
In this section we would like to suggest different future research areas based on our experiment results and observations.

In our experiment we used two different grid sizes and in the future we would suggest more research on the effect of grid sizes.
With only two different sizes we are not able to say anything about the size of the target on a large display.

For further research on techniques used for large displays and large public displays, we would suggest research on the choice of technique.
We imagine experiments and research looking at the question of choosing one technique over another for large displays and especially for walk up and use situations.

Whereas our experiment is considering push techniques, we would strongly suggest looking at pull techniques for cross device natural user interaction with large displays.
With pull techniques we imagine research investigating the opposite direction i.e. pulling information from a large display to a handheld device.
The opposite techniques of techniques which may be preferred for pushing information to a large display might not be the best choice for pulling information.

Our experiment has exclusively been concerned with some specific measures like speed and time for each technique, and we could suggest for future research that other measures be included in experiment.
This includes, but is not limited to, measures on user experience and which techniques users prefer to use for interaction with public displays.

Our experiment focused only on the interaction between mobile phones and large displays.
In the future though, the range of personal devices and ubiquitous computing will probably be much more widespread than today, and our suggestion calls for research in this area.
An example of this research is the interaction between large displays and devices such as smart watches, tables, and smart glasses.

As a final suggestion, a framework for cross device natural user interactions might help for example developers and researchers with techniques, guidelines and designs for interacting with large displays in the future.