% !TEX root = ../paper.tex
\section{Related work} \label{sec:relatedwork}
\todo[inline]{Write a related work section with reference to the literature review.}

Hamilton et al. created a framework for cross device interaction in \textit{Conductor} (2014). \todo{create reference} While their main focus here was creating cross device interaction between multiple interactive devices, never the less it could be used for interaction between large screens and mobile devices, as they themselves showed by synchronizing a map view on a tablet onto the television in the living room. They present several interaction methods and scenarios and discuss their findings of a field study they conducted with their framework. Their study expanded our knowledge of how people interact with multiple devices when presented with the possibility of cross device interaction. 

In \textit{Scroll, Tilt or Move It}, Boring et al. (2009) \todo{Create reference} investigated cross device interaction between large displays and mobile phones. More specifically, he investigated three different interaction techniques in order to continuously control a pointer on a large screen from a mobile device. \textit{Move} and \textit{Tilt}, two of the three interaction techniques, enabled faster selection time compared to the last one, \textit{Scroll}, but at the cost of higher error rates. They showed that different interaction techniques have certain strengths and weaknesses, and depending on the context and use, certain techniques can be used rather than other. 

In \textit{Shoot and Copy}, Boring et al. (2007) \todo{Create reference} explored the transfer of data from a large public display onto a mobile device. They created a method of transferring data from a large screen by using the camera on the mobile device. The user would take a picture of whatever content they were interested in, and would query the content server with the picture they took. Through visual analyses, the content server would determine what content the user was interested in and would return that content to them. They show that there is a need for enabling data exchange between mobile devices and public displays.  