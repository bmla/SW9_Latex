% !TEX root = ../paper.tex
\section{Introduction} \label{sec:introduction}
The evolution of ways people interact with the digital is noteworthy considering the short life-span of computing. How we use our devices, which devices we use, and the context in which we use them has been under transformation multiple times. From portable personal computers, originally considered mostly for specialized field applications, such accountancy, military use, or for sales representatives, which addressed mobility of a person's workspace, to modern hand-held devices which presented, their users, such degree of freedom that ultimately workspaces are starting to fade. This expansion has not only increased mobile computing, but also made it widespread. \\

As divergent devices are being adopted in different domains and context, cross-device interaction is currently becoming more important and relevant, as people take their hand-held devices with them everywhere. This ubiquitous presence of devices means that it can be leverage to enhance everyday situations in all kind of places. Imagine if a public display can morph from one-way interface device that merely show visual content to a two-way interaction device that provides more engaging and immersive experience. Given this emergence of new technology, we need to understand how different interaction techniques perform in use, e.g. in terms of how easily, quickly and accurately, or in terms of how enjoyable or satisfying it's it to interact in this way. \\

Research in the area of cross-device interaction has been trying to keep up with the changing trends, earliest examples are in the late 90's, inside ubiquitous computing, with Rekimoto's work,  he argued for, what he called, multi computed user interface and that interaction technique must overcome the boundaries among devices in multi-device settings\cite{Rekimoto:1998}.

Recent HCI research has focused on how to include in cross-device interaction an extra natural modality contributing to what should be know as cross-device natural user interaction.  Some researchers used spatial information \cite{Marquardt:2011, Marquardt:2012}, others used touch \cite{Seifert:2012}, or combined touch with air gestures \cite{Bragdon:2011} . But we still have limited understanding of how to design cross-device natural user interaction techniques and also we are missing on quantitative studies of the aforementioned.\\

Inspired by the great opportunities presented of such challenges, this paper reports on a quantitative study between four different cross-device natural user interaction techniques for data transfer between hand-held device and large displays.

We discover that out of the four techniques we developed and implemented, \swipe was the most precise and effective technique, which could be used to create more interesting and effective natural user interfaces with. We uncover that the need for precision comes at a great cost in regards to effectiveness, and should be side-stepped if possible while designing and implementing the application.  

