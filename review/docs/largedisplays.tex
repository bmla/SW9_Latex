% !TEX root = ../literature.tex
\subsection{Interacting with Large Displays}
Influenced by Weiser's paper \cite{Weiser:1991} in which he states that ubiquitous computers have to be of different sizes, each suited to a particular task. 
As real power emerges from the interaction of all the different devices, researchers started pushing the boundaries towards constructing large displays used horizontally as tabletop displays or vertically as display-walls.
 
Welnner showed the first examples of a display used as a tabletop -  Digital Desk[32] he utilized optical and acoustic finger detection on the tabletop and also played with the possibility of tactile manipulation of physical objects on the surface. Portfolio Wall was developed at the end of the 90s as a system that utilized a vertical touch-screen display by Buxton[33]. His idea was proposed as a corkboard in digital format used to share work inside the design team.

Rapid progression in the technology for displays and projections, naturally led to lowering of the marginal price, thus ``large displays have moved out of research laboratories into public spaces such as museums, libraries, plazas, and architectural facades, where they present information and enhance experiences'' [3]

We can summarize that Public displays is an application of large display therefore examining the latter without the first is a futile action.