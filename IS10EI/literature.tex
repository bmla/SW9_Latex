\documentclass[chi_draft]{sigchi}

% Use this command to override the default ACM copyright statement
% (e.g. for preprints).  Consult the conference website for the
% camera-ready copyright statement.

%% EXAMPLE BEGIN -- HOW TO OVERRIDE THE DEFAULT COPYRIGHT STRIP -- (July 22, 2013 - Paul Baumann)
% \toappear{Permission to make digital or hard copies of all or part of this work for personal or classroom use is      granted without fee provided that copies are not made or distributed for profit or commercial advantage and that copies bear this notice and the full citation on the first page. Copyrights for components of this work owned by others than ACM must be honored. Abstracting with credit is permitted. To copy otherwise, or republish, to post on servers or to redistribute to lists, requires prior specific permission and/or a fee. Request permissions from permissions@acm.org. \\
% {\emph{CHI'14}}, April 26--May 1, 2014, Toronto, Canada. \\
% Copyright \copyright~2014 ACM ISBN/14/04...\$15.00. \\
% DOI string from ACM form confirmation}
%% EXAMPLE END -- HOW TO OVERRIDE THE DEFAULT COPYRIGHT STRIP -- (July 22, 2013 - Paul Baumann)

% Arabic page numbers for submission.  Remove this line to eliminate
% page numbers for the camera ready copy
% \pagenumbering{arabic}

% Load basic packages
\usepackage{balance}  % to better equalize the last page
\usepackage{graphicx} % for EPS, load graphicx instead 
\usepackage{float}
\usepackage[caption = false]{subfig}
\usepackage[T1]{fontenc}
\usepackage{txfonts}
\usepackage{mathptmx}
\usepackage[pdftex]{hyperref}
\usepackage{color}
\usepackage{booktabs}
\usepackage{textcomp}
\usepackage[inline]{enumitem}
\usepackage{xspace}
\usepackage[table,xcdraw]{xcolor}
\usepackage{enumitem}
\usepackage[utf8]{inputenc}% Brug af æ, ø og å
% Some optional stuff you might like/need.
\usepackage{microtype} % Improved Tracking and Kerning
% \usepackage[all]{hypcap}  % Fixes bug in hyperref caption linking
\usepackage{ccicons}  % Cite your images correctly!
% \usepackage[utf8]{inputenc} % for a UTF8 editor only

%OUR OWN REF PACKAGES
\usepackage{hyperref} %doublecheck the need for this one, this tex file has a global style for URL
\usepackage{cleveref}


% If you want to use todo notes, marginpars etc. during creation of your draft document, you
% have to enable the "chi_draft" option for the document class. To do this, change the very first
% line to: "\documentclass[chi_draft]{sigchi}". You can then place todo notes by using the "\todo{...}"
% command. Make sure to disable the draft option again before submitting your final document.
\usepackage{soul}
\usepackage{todonotes}
\newcommand{\hlfix}[2]{\hl{#1}\todo{#2}}
%Example \hlfix{wrong text}{fix text}~continues here.


% Paper metadata (use plain text, for PDF inclusion and later
% re-using, if desired).  Use \emtpyauthor when submitting for review
% so you remain anonymous.
\def\plaintitle{Comparing 4 Push Interaction techniques for Cross-device Natural User Interactions with Public Displays and Mobile devices}
\def\plainauthor{First Author, Second Author, Third Author,
  Fourth Author, Fifth Author, Sixth Author}
\def\emptyauthor{}
\def\plainkeywords{Interaction Techniques; Cross-Device Interaction; Natural User Interaction; Kinect; Mid-air Gestures; Public Space; Public Displays; Large Displays}
\def\plaingeneralterms{Documentation, Standardization}

% llt: Define a global style for URLs, rather that the default one
\makeatletter
\def\url@leostyle{%
  \@ifundefined{selectfont}{
    \def\UrlFont{\sf}
  }{
    \def\UrlFont{\small\bf\ttfamily}
  }}
\makeatother
\urlstyle{leo}

% To make various LaTeX processors do the right thing with page size.
\def\pprw{8.5in}
\def\pprh{11in}
\special{papersize=\pprw,\pprh}
\setlength{\paperwidth}{\pprw}
\setlength{\paperheight}{\pprh}
\setlength{\pdfpagewidth}{\pprw}
\setlength{\pdfpageheight}{\pprh}

% Make sure hyperref comes last of your loaded packages, to give it a
% fighting chance of not being over-written, since its job is to
% redefine many LaTeX commands.
\definecolor{linkColor}{RGB}{6,125,233}
\hypersetup{%
  pdftitle={\plaintitle},
% Use \plainauthor for final version.
%  pdfauthor={\plainauthor},
  pdfauthor={\emptyauthor},
  pdfkeywords={\plainkeywords},
  bookmarksnumbered,
  pdfstartview={FitH},
  colorlinks,
  citecolor=black,
  filecolor=black,
  linkcolor=black,
  urlcolor=linkColor,
  breaklinks=true,
  hypertexnames=false
}

% create a shortcut to typeset table headings
% \newcommand\tabhead[1]{\small\textbf{#1}}

% End of preamble. Here it comes the document.
\begin{document}

\title{\plaintitle}

\numberofauthors{2}
\author{%
	\alignauthor{Ivan S. Penchev\\
		\email{Aalborg University\\
			ipench14@student.aau.dk}}\\
	\alignauthor{Elias Ringhauge\\
		\email{Aalborg University\\
			eringh10@student.aau.dk}}\\
}

\maketitle

% !TEX root = ../paper.tex
\begin{abstract}
\todo[inline]{Write abstract}

\todo{Add section numbers to each section because of references to certain sections of the paper}

\end{abstract}


\category{H.5.m.}{Information Interfaces and Presentation
  (e.g. HCI)}{Miscellaneous} \category{See
  \url{http://acm.org/about/class/1998/} for the full list of ACM
  classifiers. This section is required.}{}{}

\keywords{\plainkeywords}

% !TEX root = ../literature.tex
\section{Introduction}
Weiser\cite{Weiser:1991} points out:
 
{\em``Ubiquitous computing names the third wave in computing, just now beginning. First were mainframes, each shared by lots of people. Now we are in the personal computing era, person and machine staring uneasily at each other across the desktop. Next comes ubiquitous computing, or the age of calm technology, when technology recedes into the background of our lives.''}.

Weiser's vision of ubiquitous computing, is over 20 years old, includes the ubiquity availability of computers that are preferably not distinguishable from everyday objects. He also talks about "calm technology", which is this technology that resides in the periphery and plays a non-dominant role in a user's life. The perception of devices plays a major role in Weiser's vision, however the other side of that is availability of devices and also computing power, which is important for cross-device interaction. As so we could contend and regard that the latest incarnation of Weiser's vision is cross-device interaction, where ideally joining several devices would lead to single seamless, and natural user interaction, flexible, and not restricted to a few configurations\cite{Radle:2015}.
In order for this interaction to happen we need to remain as close as possible to the real world and have multimodality in mind, as Jain et al. states: ``human interaction with the world is multi-modal, and rich multi-modal interaction is part of what defines a natural experience.''\cite{Jain:2011}. 
This corresponds well witth the way Wigdor and Wixon defines NUI , where "Natural" is about how the users feel and what they do when using a product. The products must mirror the user's capabilities and meet their needs, but the trick is to help the users feel comfortable about the product without the need for much practice\cite{Wigdor:2011}. This becomes important for the design of product that has a non-dominant role in a user's life as the user should not be using a lot of time for adapting to the products.   
Ideally we would see a combination of multiple modalities, for instance gestures, augmented reality, touch, voice recognition etc. 
Researchers have made and published breakthroughs and designers constructed innovation design in each modality individually, however there has only been little research and work in combining them \cite{Jain:2011}. \\

It is possible for the existence of untapped opportunities in the integration of multiple natural user interaction modalities for enriching the cross-device experience with hand-held devices and large public displays. \\

In this paper we review the research that has been done within interaction with large displays. 
This has been done to map the current understanding and practices, thereby helping future researchers, within this field, to gain a perspective as well as identifying possible untapped opportunities for future work.
%% !TEX root = ../literature.tex
\section{State of art}
We present an overview of research in interaction techniques for cross device interaction, with an emphasis on the natural user interface.
By looking at Public Space we found the following topics in HCI: Cross-device Interaction, Large Displays, Multi-device Interaction, Natural User Interface, Proxemics, Social Interaction, Tabletop Interaction, and Tangible Interaction. 
We used these topics for our inquiry in ACM DigitalLibrary, SpringerLink and Microsoft Research, which matched a total of 56 papers. 
To narrow down our focus we selected papers that matched a set of strict criteria: (1) The paper should contain one or more interaction techniques, (2) It is important that the techniques are evaluated using either interviews, surveys, field studies or any other research method.
The final set contained 34 literature sources which we further examined, by writing a short summary for each paper which included ``title, author's thesis (aim of paper), objectives (goals, summaries), methodologies used, findings, conclusion, and keywords''. 
Using applied thematic analysis different themes were induced on the summaries, which we later categorized into 5 themes: interaction with large displays, interaction with public displays, natural user interactions, cross-device interactions, cross-device natural user interactions.
The identified themes and the specific papers correlated to them are presented in Table 1, with full citations in the references section of this paper. 

\begin{table*}[t]
\centering
\begin{tabular}{@{}ll@{}}
\toprule
Theme & Paper \\ \midrule
Interacting with Large displays      &      [1] [3] [32][33] \\
Interacting with Public displays      &       [4] [5] [6] [7] [8] [10] [11] [12] [34] \\
Natural user interactions      &      [9] [13] [14] [15] [16] [17] [18] [19] [30] [31] \\
Cross-device interactions      &      [20] [21] [22] [23] [24] [25] \\
Cross-device natural user interactions      &      [26] [27] [28] [29] \\ \bottomrule
\end{tabular}
\caption{Among the final 34 papers found, we identified 5 themes.}
\label{table:themes}
\end{table*}
%\section{Conclusion}\label{Sec:Conclusion}
In this paper we presented an overview of HCI research within interacting with large displays. We reviewed 34 papers from the last decade in detail. A picture (\Cref{fig:litreview}) was drawn about the relation of large displays, public displays, cross-device interaction and natural user interaction. Cross device interaction and natural user interaction was found to be a subset of large- and public displays. Based on further examination of these areas in research, we identified cross-device natural user interaction as well as some opportunities and shortcomings that we used to suggest possible future research areas.\\
In the future we would like to see quantitative research within the field of cross-device natural user interaction, as well as use of the produced statistically solid results as a guide for the design of further applications for cross-device interactions and large displays.

%\subsection{References and Citations}
%
%Use a numbered list of references at the end of the article, ordered
%alphabetically by last name of first author, and referenced by numbers
%in
%brackets~\cite{acm_categories,ethics,Klemmer:2002:WSC:503376.503378}.
%Your references should be published materials accessible to the
%public. Internal technical reports may be cited only if they are
%easily accessible (i.e., you provide the address for obtaining the
%report within your citation) and may be obtained by any reader for a
%nominal fee. Proprietary information may not be cited. Private
%communications should be acknowledged in the main text, not referenced
%(e.g., ``[Borriello, personal communication]'').
%
%References should be in ACM citation format:
%\url{http://acm.org/publications/submissions/latex_style}. This
%includes citations to internet
%resources~\cite{acm_categories,cavender:writing,CHINOSAUR:venue,psy:gangnam}
%according to ACM format, although it is often appropriate to include
%URLs directly in the text, as above.


% Use a numbered list of references at the end of the article, ordered
% alphabetically by first author, and referenced by numbers in
% brackets~\cite{ethics, Klemmer:2002:WSC:503376.503378,
%   Mather:2000:MUT, Zellweger:2001:FAO:504216.504224}. For papers from
% conference proceedings, include the title of the paper and an
% abbreviated name of the conference (e.g., for Interact 2003
% proceedings, use \textit{Proc. Interact 2003}). Do not include the
% location of the conference or the exact date; do include the page
% numbers if available. See the examples of citations at the end of this
% document. Within this template file, use the \texttt{References} style
% for the text of your citation.

% Your references should be published materials accessible to the
% public.  Internal technical reports may be cited only if they are
% easily accessible (i.e., you provide the address for obtaining the
% report within your citation) and may be obtained by any reader for a
% nominal fee.  Proprietary information may not be cited. Private
% communications should be acknowledged in the main text, not referenced
% (e.g., ``[Robertson, personal communication]'').

% Balancing columns in a ref list is a bit of a pain because you
% either use a hack like flushend or balance, or manually insert
% a column break.  http://www.tex.ac.uk/cgi-bin/texfaq2html?label=balance
% multicols doesn't work because we're already in two-column mode,
% and flushend isn't awesome, so I choose balance.  See this
% for more info: http://cs.brown.edu/system/software/latex/doc/balance.pdf
%
% Note that in a perfect world balance wants to be in the first
% column of the last page.
%
% If balance doesn't work for you, you can remove that and
% hard-code a column break into the bbl file right before you
% submit:
%
% http://stackoverflow.com/questions/2149854/how-to-manually-equalize-columns-
% in-an-ieee-paper-if-using-bibtex
%
% Or, just remove \balance and give up on balancing the last page.
%
\balance{}

% REFERENCES FORMAT
% References must be the same font size as other body text.
\bibliographystyle{SIGCHI-Reference-Format}
\bibliography{literature}

\end{document}

%%% Local Variables:
%%% mode: latex
%%% TeX-master: t
%%% End:
