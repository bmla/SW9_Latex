% !TEX root = ../literature.tex
\subsection{Interacting with Large Displays}
Weiser states that ubiquitous computers have to be of different sizes, each suited to a particular task. Weiser and his coleagues have build small and large displays and encoruage the reader to consider the amount of sourrounding items that displays information, such as labels, bookspines, thermostats, etc. He states that the items \emph{``leads to our goals for initially deploying the hardware of embodied virtuality''}.\cite{Weiser:1991} 
%shudn't this be Weiser et al ?

Welnner showed the first examples of a display used as a tabletop \emph{Digital Desk}, he utilized optical and acoustic finger detection on the tabletop, and also played with the possibility of tactile manipulation of physical objects on the surface.\cite{Wellner:1993} \emph{Portfolio Wall} was developed at the end of the 90s as a system that utilized a vertical touch-screen display by Buxton\cite{Buxton:2000}. His idea was proposed as a corkboard in digital format used to share work inside the design team.\\ % not sure about ``corkboard'' and can we write something more to it ? Are the two refferences et al or not? 

Czerwinski et al. made a study on the productivity of using large displays in a office. They hypothesize that there will be a cognitive load advantage and a reduction in window management by using a large display. They build a novel 42'' wide surface called \emph{DSharp} by combining 3 projectors onto a curved Plexiglas panel that gives an almost seamless display with a 3072 x 768 resolution. In a user study they compare a 15'' flat panel display with their DSharp display to deduce if there was a significant performance advantage by performing tasks on the larger display. Their results showed a significant faster task completion time on the larger display corresponding to a 9\% increased productivity and that 14 of their 15 participants preferred the large display.

%This article style seems not to use indents. Be carefull with the citation, Is it obvious that the two paragraphs are together. Is starting with "However'' bad?
However they also discovered some usability issues for the large display during their test, such as the distance between the display and user being too small as some users would prefer to back up and interact with the display at a distance. They also discovered that the software design didn't fit the usage of multiple displays, as the users mentioned that the amount of navigation and losing the cursor on the display were the most onerous problems with the large display.
Czerwinski et al. suggests that the taskbar is stretched across all the displays, and that the start menu and notifications comes up on the display which the user has its attention on. Likewise they suggest to improve the  visualization of the cursor to make it easier to locate on a large display.\cite{Czerwinski:2003}\\
%Toward Characterizing the Productivity Benefits of Very Large Displays - Mary Czerwinski, Greg Smith, Tim Regan, Brian Meyers, George 
%Robertson and Gary Starkweather 
%Microsoft Research, One Microsoft Way, Redmond, WA, 98052, USA  - 2003
%\todo{Make and add citation}
% Perhaps some Urban HCI

Rapid progression in the technology for displays and projections, naturally led to lowering of the marginal price, thus \emph{``large displays have moved out of research laboratories into public spaces such as museums, libraries, plazas, and architectural facades, where they present information and enhance experiences''} \cite{Hinrichs:2013:IPD:2478559.2478965}.  %what do  we want to reflect upon from this?  
While we know from Czerwinski et al.'s findings  that large displays provide an increased productivity, we also know that there are usability challenges such as tracking the cursor and lack of design that supports multimonitor displays. Futhermore Czerwinski et al.'s findings also suggest that users wants to interact with large displays from a distance, which both brings new challenges as well as oppotunities. We have examined the large display in an office and lab context that  doesn't take the complications of public space into consideration; this will be done in the next section.
%We can summarize that Public displays is an application of large display therefore examining the latter without the first is a futile action. % Lets put this inside public display.