% !TEX root = ../paper.tex
\subsection{Results of Survey Easy of Use}
Our questionnaire was based on USE, which used Likert scale \ref*{sec:expdesign}, when encoding the data we did it as continuous variable, as such \emph{"strongly disagree"} got a value of 1, and \emph{"strongly agree"} a value ot 7. After that the cumulative value per technique, based on the different questions, was calculated, the data was ploted and presented in \ref{fig:surveyResult}. 
\begin{figure}[H]
	{\includegraphics[width = 1\columnwidth , height = 6cm ]{images/survey-data.png}} 
	\caption{
		Cumulative values of survey questions per technique
	}
	\label{fig:surveyResult}
\end{figure}

A One-Way MANOVA was applied ($F(18, 574.66)=5.118,\ p<0.000$) and only the differences that were not statistically different will be mentioned. 
Therefore, the unspecified combinations in the following list, a result of the post hoc test, are statistically significant: 
\begin{enumerate*}[label=\itshape\arabic*\upshape)]
	\item{``It is easy to use'' there is no statistical difference between \throw, \tilt, and \pinch;}
	\item{``Using it is effortless'' there is no statistical difference between \throw, \tilt, and \pinch;}
	\item{``It is easy to learn to use'' there is no statistical difference between \tilt, and \throw;}
	\item{``I can use it successfully every time'' there is no statistical difference between \tilt, and \pinch;}
	\item{``I quickly became skillful with it'' there is no statistical difference between \pinch, and \tilt;}
	\item{``I learned how to use it quickly'' there is no statistical difference between \throw, \tilt, and \pinch.}
\end{enumerate*}