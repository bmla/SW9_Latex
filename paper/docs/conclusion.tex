% !TEX root = ../paper.tex
\section{Conclusion}
We have presented 4 different interaction techniques and compared them to each other, in order to determine their strengths and weaknesses.
We also examined the effect of having larger vs. smaller targets on the screen, in order to simulate the need for either less or increased precision in applications and see what kind of effects it had on time per target and hit ratio per target. 

We discovered that out of the four techniques, \swipe, \throw, \tilt and \pinch, \swipe was the most precise and effective technique. We also showed that \swipe was the technique users had the most positive attitude towards. 
\swipe could be used to create more formal applications, were the need for effectiveness and precision are critical it's success, such as a presentation tool or a professional data exchange tool. 

More research should be done towards more engaging and fun interaction techniques.
\pinch showed some potential in this department, with quite a few users showing interest and excitement while using this technique.
Research could lead to techniques that could be used to create more entertaining and engaging applications, where their use context was not as formal. This could be for example games or multimedia control applications. 

We also uncovered that the need for precision comes at a great cost for effectiveness. 
If possible, developers should try to side-step this need by either making targets large or error friendly.
This could be achieved by developing methods of accurately determining the users intended target.
"Snapping" behavior could also be implemented to great effect, where if the users managed to hover close the target, the pointer would snap onto the target, decreasing the need for accuracy. 