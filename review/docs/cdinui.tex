% !TEX root = ../literature.tex
%\subsection{Cross-Device Interaction Natural User Interface} - Merge no publlic space, Ganbaru desu!
Infusing the two fields could offer new concepts and potential ideas, as such an inquiry of the current literature would be beneficial. 
An illustrative example, on transitioning from one to another area, would be Greenberg et al. work \cite{Greenberg:2011} who expand on a concept from Psychology, which states that different distance between people can explain their relation, by adapting it for use in HCI, they do so by concerning distance, not only between people, but between all objects such as people, electronic devices and analogue objects. Their main contribution is in defining five proxemic dimensions: \emph{Distance, Orientation, Movement, Identity} and \emph{Location}. This dimensions can be used to help devices, in a system ecology , to recognize their current context, distance to other people or devices, and can help creating interconnectivity between entities. 

Marquardt work, that proposes to use spatial information - proxemics {\em``[...] to mediate people's interactions with digital devices, such as large digital surfaces or portable personal devices.''} \cite{Marquardt:2011}, gives an illustration of the use within cross-device. He presents design of development tools for programmers creating proxemic-aware systems, his aim is {\em``to create techniques that allow people to seamlessly and naturally connect to and interact with the increasing number of digital devices.''} \cite{Marquardt:2011}. 


Marquardt et. al. later expand this idea by increasing the number of dimensions i.e. he augments proxemic principles with theories of F-formation and micro-mobility, his creation is a system that uses combination of Kinect sensors and accelerometer data, which was combined to determine the locations and orientation of users in an instrumented environment . This information was used to facilitate a number of real-time interactions between user devices, as the author writes {\em``[...]a system that explores cross-device interaction using two sociological constructs. First, F-formations concern the distance and relative body orientation among multiple users, which indicate when and how people position themselves as a group. Second, micro-mobility describes how people orient and tilt devices towards one another to promote fine-grained sharing during co-present collaboration.''} \cite{Marquardt:2012}. Combining these two aspects, new techniques of cross-device interaction with emphasis on fluid and smooth communication can be designed argues the author. For instance, tilting a device by a small angle may trigger an information sharing process with other devices within proximity\\ 


Proxemics is one of the ways to create cross-device natural user interaction.
Bragdon et al. propose using a system that combines touch + air gesture hybrid interactions.
They aim to design, implement and test a system that allows a group of users to interact using air gestures and touch gestures. 
The purpose is to increase control, support democratic access, and share items across multiple personal devices such as smartphones and laptops where the {\em``primary design goal is fluid, democratic sharing of content on a common display.''} \cite{Bragton:2011}. This method enables access, control and sharing of information through several different devices such as multi-touch screen, mobile touch devices, and Microsoft Kinect sensors. In a formative study, professional developers were positive about the interaction design, and most felt that pointing with hands or devices and forming hand postures are socially acceptable. 