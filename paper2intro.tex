%\section*{Paper I: Literature Review}
\addcontentsline{toc}{section}{Paper 2: Research Paper}




\drop=0.1\textheight
\centering
\vspace*{\baselineskip}
\rule{\textwidth}{1.6pt}\vspace*{-\baselineskip}\vspace*{2pt}
\rule{\textwidth}{0.4pt}\\[\baselineskip]
{\LARGE PAPER 2: \\ RESEARCH \\[0.3\baselineskip] PAPER}\\[0.2\baselineskip]
\rule{\textwidth}{0.4pt}\vspace*{-\baselineskip}\vspace{3.2pt}
\rule{\textwidth}{1.6pt}\\[\baselineskip]
\scshape
{ \large Comparison of Push Techniques for Cross-Device Interaction Between Mobile Devices and Large Displays } \par
\vspace*{2\baselineskip}
%Edited by \\[\baselineskip]
\vspace*{2\baselineskip}
\vspace*{2\baselineskip}
\vspace*{2\baselineskip}



\begin{newab}
	{In recent years, research into cross-device interaction techniques has increased. Much of this research focuses on interaction between mobile devices and large displays. We contribute to this body of knowledge with an empirical comparison of four different push techniques - pinch, swipe, throw,  and tilt for interaction between mobile devices and large displays. We report on Efficiency, Success rate and Accuracy. We also present the ease of use of techniques as perceived by users. We show that swipe was the most effective in terms of success rate, efficiency and accuracy. Furthermore, swipe gathered the highest score, in regards to ease of use, by users. Participants also reported that pinch was the most fun to use.}\par
\end{newab}

	\vspace*{2\baselineskip}
	\vspace*{2\baselineskip}
	\vspace*{2\baselineskip}
	\vspace*{2\baselineskip}
{\scshape 2016} \\
{\large AALBORG UNIVERSITY}\par

