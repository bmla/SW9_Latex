% !TEX root = ../paper.tex
%http://sphweb.bumc.bu.edu/otlt/MPH-Modules/BS/BS704_HypothesisTesting-ANOVA/BS704_HypothesisTesting-Anova_print.html
\section{Results}
We will now present the results that we achieved through out our experiment and also how they were achieved. We will first present our findings in respect to success rate, then in respect to efficiency, and finally in respect to accuracy. Finally, we will look at the questionnaires and show the significant findings there. We will discuss these results later, in the \nameref{sec:discussion} section.  

\subsection{Success Rate}
In this section we will present the results related to the success of hitting a target.
We will perform an analysis of the success rate and the effect of each technique with respect to grid sizes.

The success rate's mean and standard deviation for each technique for small and large grid size can be seen in \Cref{tab:meanHitTechniqueSmall} and \Cref{tab:meanHitTechniqueLarge} respectively.

\begin{table}[H]
	\centering
	\textbf{Small Grid Success Rate}\\[4pt]
	\begin{tabular}{|c|c|c|c|c|}
			\hline
			\rowcolor[HTML]{9B9B9B} 
			& \textbf{Pinch} & \textbf{Swipe} & \textbf{Throw} & \textbf{Tilt} \\
			\rowcolor[HTML]{9B9B9B} 
			 & (n = 477) & (n = 477) & (n = 477) & (n = 477) \\ \hline
			Mean & 0.65         & 0.91          & 0.83          & 0.58         \\ \hline
			Std. Dev. & 0.48 & 0.29 & 0.37 & 0.49 \\ \hline
	\end{tabular}
	\caption{Mean hit  and standard deviation for each technique per target for small grid.}
	\label{tab:meanHitTechniqueSmall}
\end{table}

\begin{table}[H]
	\centering
	\textbf{Large Grid Success Rate}\\[4pt]
	\begin{tabular}{|c|c|c|c|c|}
			\hline
			\rowcolor[HTML]{9B9B9B} 
			& \textbf{Pinch} & \textbf{Swipe} & \textbf{Throw} & \textbf{Tilt} \\
			\rowcolor[HTML]{9B9B9B} 
			 & (n = 477) & (n = 477) & (n = 477) & (n = 477) \\ \hline
			Mean & 0.78         & 0.97          & 0.94          & 0.78         \\ \hline
			Std. Dev. & 0.41 & 0.18 & 0.25 & 0.41 \\ \hline
	\end{tabular}
	\caption{Mean hit  and standard deviation for each technique per target for large grid.}
	\label{tab:meanHitTechniqueLarge}
\end{table}

In order to see the effect of each technique on the hit ratio per target for the different grid sizes, we performed two different one-way ANOVA's, where we split the data between the two different grid sizes.
We then performed a post-hoc pairwise LSD test to see where the significant difference were.
 
For the the small grid, we got $(p<0.001)$, $(F(2.722, 1295.674)=62.754)$, (Greenhouse-Geisser correction: 0.907).
The pairwise test showed that all techniques were significantly different. \pinch and \swipe had $(p < 0.001)$, \pinch and \throw $(p <0.001)$, \pinch and \tilt $(p = 0.031)$, \swipe and \throw $(p=0.001)$, \swipe and \tilt $(p < 0.001)$ and finally, \throw and \tilt had $(p<0.001)$. 

For the large grid, we got $(p<0.001)$, $(F(2.472, 1176.749)=42.773)$, (Greenhouse-Geisser correction: 0.824).
The pairwise test showed that all of the techniques, with the exception of \pinch and \tilt, were statistically different from each other. The results were as following: \pinch and \swipe $(p<0.001)$, \pinch and \throw $(p<0.001)$, \pinch and \tilt $(p=1.000)$, \swipe and \throw $(p=0.025)$, \swipe and \tilt $(p<0.001)$, and finally \throw and \tilt $(p<0.001)$.

\subsection{Efficiency}
In this section we present the efficiency results which defines the amount of time spent performing a technique.
We perform an analysis of the efficiency and the effect of each technique with respect to grid sizes.

The four interaction techniques, \pinch, \swipe, \throw and \tilt were completed 18 times per participant. 
For each grid size, each technique was performed 477 times. 
\Cref{tab:meanTimesTechniqueSmall} and \Cref{tab:meanTimesTechniqueLarge} shows the mean time per target for each of the techniques as well as their standard deviation for both grid sizes. 
\begin{table}[H]
	\centering
	\textbf{Small Grid Efficiency}\\[4pt]
	\begin{tabular}{|c|c|c|c|c|}
		\hline
		\rowcolor[HTML]{9B9B9B} 
		 & \textbf{Pinch} & \textbf{Swipe} & \textbf{Throw} & \textbf{Tilt} \\ 
		 \rowcolor[HTML]{9B9B9B} 
		 & (n = 477) & (n = 477) & (n = 477) & (n = 477) \\ \hline
		Mean & 9.23  sec         & 6.41 sec          & 7.73 sec          & 6.67 sec         \\ \hline
		Std. Dev. & 6.48 sec & 4.49 sec & 6.6 sec & 4.49 sec \\ \hline
	\end{tabular}
	\caption{Mean time and standard deviation for each technique per target for small grid.}
	\label{tab:meanTimesTechniqueSmall}
\end{table}

\begin{table}[H]
	\centering
	\textbf{Large Grid Efficiency}\\[4pt]
	\begin{tabular}{|c|c|c|c|c|}
		\hline
		\rowcolor[HTML]{9B9B9B} 
		 & \textbf{Pinch} & \textbf{Swipe} & \textbf{Throw} & \textbf{Tilt} \\
		 \rowcolor[HTML]{9B9B9B}
		 & (n = 477) & (n = 477) & (n = 477) & (n = 477) \\ \hline
		Mean & 8.09  sec         & 5.01 sec          & 6.42 sec          & 5.33 sec         \\ \hline
		Std. Dev. & 6.6 sec & 2.66 sec & 5.43 sec & 3.04 sec \\ \hline
	\end{tabular}
	\caption{Mean time and standard deviation for each technique per target for large grid.}
	\label{tab:meanTimesTechniqueLarge}
\end{table}

A one way, repeated measure ANOVA revealed, with a Greenhouse-Geisser correction (0.874), that the difference in time per target between the different techniques is in fact a significant($F(2.622, 2498.37)=67.192$, $p<0.001$). A post hoc pair-wise test using a LSD correction revealed that \pinch and \swipe($p<0.001$), \pinch and \tilt ($p<0.001$), \pinch and \throw ($p<0.001$), \swipe and \throw ($p<0.001$), and \throw and \tilt ($p<0.001$) were all significantly different from each other. However, \swipe and \tilt were not ($p=0.092$)

We then examined the effect of grid size on time. Each participant had a equal mixture of small and large grid targets. This means that the four techniques had 9 targets of each size; four large and four small. Every test had 36 targets of each size, totaling 1908 targets of each size, through out all tests. \Cref{tab:meanTimesSize} shows the mean time per target and standard deviation for each grid size. 

To determine the effect of grid size on target hit time, we applied another one way, repeated measures ANOVA($F(1,1907) = 77.587$, $p<0.001$). This means that the grid size has a significant effect on time when hitting the target. 

\subsection{Accuracy}
Finally, we will perform an analysis of the accuracy and the effect of each technique with respect to grid sizes.
Here, we took three different measures of accuracy; the distance between where the user hit and the target cell and the $x$ and $y$ axis independently as well. 
These were all measured in pixels.
This was because there were signs that certain techniques would usually miss in a specific direction, and we wanted to see if the data supported that. 
An overview of the data can be in \Cref{tab:distanceSmall} and \Cref{tab:distanceLarge}.

\begin{table}[H]
	\centering
	\begin{tabular}{|c|c|c|c|c|}
		\hline
		\rowcolor[HTML]{9B9B9B} 
		& \textbf{Pinch} & \textbf{Swipe} & \textbf{Throw} & \textbf{Tilt} \\
		\rowcolor[HTML]{9B9B9B} 
		& (n = 477) & (n = 477) & (n = 477) & (n = 477) \\ \hline
		Mean Dist. & 75.97 & 5.40 & 18.60          & 76.51         \\ \hline
		SD. Dist. & 176.15 & 47.81 & 95.29 & 177.45 \\ \hline
		Mean XD. & 54.33 & 3.78 & 10.32 & 49.23 \\ \hline
		SD. XD. & 140.87 & 40.15 & 81.84 & 151.36 \\ \hline
		Mean YD. & 42.90 & 2.00 & 8.00 & 39.33 \\ \hline
		SD. YD. & 110.30 & 26.16 & 49.37 & 99.82 \\ \hline
	\end{tabular}
	\caption{Mean and standard deviation for the distance, distance on the X-Axis(XD) and distance on the Y-Axis(YD) for each technique per target for small grid.}
	\label{tab:distanceSmall}
\end{table}

\begin{table}[H]
	\centering
	\begin{tabular}{|c|c|c|c|c|}
		\hline
		\rowcolor[HTML]{9B9B9B} 
		& \textbf{Pinch} & \textbf{Swipe} & \textbf{Throw} & \textbf{Tilt} \\
		\rowcolor[HTML]{9B9B9B} 
		& (n = 477) & (n = 477) & (n = 477) & (n = 477) \\ \hline
		Mean Dist. & 75.41 & 2.29 & 12.37 & 59.88         \\ \hline
		SD. Dist. & 187.58 & 38.82 & 81.36 & 172.22 \\ \hline
		Mean XD. & 55.32 & 1.88 & 10.32 & 49.23 \\ \hline
		SD. XD. & 159.88 & 31.74 & 74.01 & 157.35 \\ \hline
		Mean YD. & 37.41 & 1.14 & 4.33 & 22.86 \\ \hline
		SD. YD. & 104.18 & 22.37 & 34.21 & 73.75 \\ \hline
	\end{tabular}
	\caption{Mean and standard deviation for the distance, distance on the X-Axis(XD) and distance on the Y-Axis(YD) for each technique per target for large grid.}
	\label{tab:distanceLarge}
\end{table} 

We performed three one way ANOVA's with all 3 different measures of sizes.
For distance, we got $(p<0.001)$ $(F(4.768, 2269.348) = 30.636)$ (Greenhouse-Geisser correction of 0.681).
For distance on the X-axis, we got $(p<0.001)$, $(F(4.788,2278.99) = 21.984)$ (Greenhouse-Geisser correction of 0.684).
Finally, for the distance on the Y-axis, we got $(p<0.001)$, $(F(248987.377, 8612.168)=28.911)$ (Greenhouse-Geisser correction of 0.624).

We then performed a post hoc LSD test to see where the significant differences were located. 
