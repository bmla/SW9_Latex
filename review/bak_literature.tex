% This is "sig-alternate.tex" V2.1 April 2013
% This file should be compiled with V2.5 of "sig-alternate.cls" May 2012
%
% This example file demonstrates the use of the 'sig-alternate.cls'
% V2.5 LaTeX2e document class file. It is for those submitting
% articles to ACM Conference Proceedings WHO DO NOT WISH TO
% STRICTLY ADHERE TO THE SIGS (PUBS-BOARD-ENDORSED) STYLE.
% The 'sig-alternate.cls' file will produce a similar-looking,
% albeit, 'tighter' paper resulting in, invariably, fewer pages.
%
% ----------------------------------------------------------------------------------------------------------------
% This .tex file (and associated .cls V2.5) produces:
%       1) The Permission Statement
%       2) The Conference (location) Info information
%       3) The Copyright Line with ACM data
%       4) NO page numbers
%
% as against the acm_proc_article-sp.cls file which
% DOES NOT produce 1) thru' 3) above.
%
% Using 'sig-alternate.cls' you have control, however, from within
% the source .tex file, over both the CopyrightYear
% (defaulted to 200X) and the ACM Copyright Data
% (defaulted to X-XXXXX-XX-X/XX/XX).
% e.g.
% \CopyrightYear{2007} will cause 2007 to appear in the copyright line.
% \crdata{0-12345-67-8/90/12} will cause 0-12345-67-8/90/12 to appear in the copyright line.
%
% ---------------------------------------------------------------------------------------------------------------
% This .tex source is an example which *does* use
% the .bib file (from which the .bbl file % is produced).
% REMEMBER HOWEVER: After having produced the .bbl file,
% and prior to final submission, you *NEED* to 'insert'
% your .bbl file into your source .tex file so as to provide
% ONE 'self-contained' source file.
%
% ================= IF YOU HAVE QUESTIONS =======================
% Questions regarding the SIGS styles, SIGS policies and
% procedures, Conferences etc. should be sent to
% Adrienne Griscti (griscti@acm.org)
%
% Technical questions _only_ to
% Gerald Murray (murray@hq.acm.org)
% ===============================================================
%
% For tracking purposes - this is V2.0 - May 2012

\documentclass[10pt]{sig-alternate-05-2015}

\usepackage{hyperref}
\usepackage{cleveref}
\usepackage[backend=biber,style=numeric,sorting=nyt]{biblatex}
\usepackage{booktabs}

\usepackage{xcolor}
\newcommand\todo[1]{\textcolor{red}{#1}}

\usepackage{graphicx}
\addbibresource{Literature.bib}

\begin{document}

% Copyright
%\setcopyright{acmcopyright}
%\setcopyright{acmlicensed}
%\setcopyright{rightsretained}
%\setcopyright{usgov}
%\setcopyright{usgovmixed}
%\setcopyright{cagov}
%\setcopyright{cagovmixed}
\setcopyright{rightsretained}

% DOI
%\doi{10.475/123_4}

% ISBN
%\isbn{123-4567-24-567/08/06}

%Conference
%\conferenceinfo{PLDI '13}{June 16--19, 2013, Seattle, WA, USA}

%\acmPrice{\$15.00}

%
% --- Author Metadata here ---
%\conferenceinfo{WOODSTOCK}{'97 El Paso, Texas USA}
%\CopyrightYear{2007} % Allows default copyright year (20XX) to be over-ridden - IF NEED BE.
%\crdata{0-12345-67-8/90/01}  % Allows default copyright data (0-89791-88-6/97/05) to be over-ridden - IF NEED BE.
% --- End of Author Metadata ---
%\permission{Permission to make digital or hard copies of all or part of
%this work for personal or classroom use is granted without fee provided 
%that copies are not made or distributed for profit or commercial advantage 
%and that copies bear this notice and the full citation on the first page. 
%Copyrights for components of this work owned by others than the author(s) 
%must be honored. Abstracting with credit is permitted. To copy otherwise, 
%or republish, to post on servers or to redistribute to lists, requires 
%prior specific permission and/or a fee. Request permissions from 
%Permissions@acm.org.}
%%%%%%%%%%%%%%%%%%
%TITLE
%%%%%%%%%%%%%%%%%%
\title{The State of The Art of Interacting with Large Displays}
%\subtitle{[Extended Abstract]



%\title{Alternate {\ttlit ACM} SIG Proceedings Paper in LaTeX
%Format\titlenote{(Produces the permission block, and
%copyright information). For use with
%SIG-ALTERNATE.CLS. Supported by ACM.}}
%\subtitle{[Extended Abstract]
%\titlenote{A full version of this paper is available as
%\textit{Author's Guide to Preparing ACM SIG Proceedings Using
%\LaTeX$2_\epsilon$\ and BibTeX} at
%\texttt{www.acm.org/eaddress.htm}}}
%
% You need the command \numberofauthors to handle the 'placement
% and alignment' of the authors beneath the title.
%
% For aesthetic reasons, we recommend 'three authors at a time'
% i.e. three 'name/affiliation blocks' be placed beneath the title.
%
% NOTE: You are NOT restricted in how many 'rows' of
% "name/affiliations" may appear. We just ask that you restrict
% the number of 'columns' to three.
%
% Because of the available 'opening page real-estate'
% we ask you to refrain from putting more than six authors
% (two rows with three columns) beneath the article title.
% More than six makes the first-page appear very cluttered indeed.
%
% Use the \alignauthor commands to handle the names
% and affiliations for an 'aesthetic maximum' of six authors.
% Add names, affiliations, addresses for
% the seventh etc. author(s) as the argument for the
% \additionalauthors command.
% These 'additional authors' will be output/set for you
% without further effort on your part as the last section in
% the body of your article BEFORE References or any Appendices.

%%%%%%%%%%%%%%%%%%
%AUTHOR LISTING
%%%%%%%%%%%%%%%%%%
\numberofauthors{5} %  in this sample file, there are a *total*
% of EIGHT authors. SIX appear on the 'first-page' (for formatting
% reasons) and the remaining two appear in the \additionalauthors section.
%
\author{
% You can go ahead and credit any number of authors here,
% e.g. one 'row of three' or two rows (consisting of one row of three
% and a second row of one, two or three).
%
% The command \alignauthor (no curly braces needed) should
% precede each author name, affiliation/snail-mail address and
% e-mail address. Additionally, tag each line of
% affiliation/address with \affaddr, and tag the
% e-mail address with \email.
%
% 1st. author
\alignauthor
Jeni Paay\\%\titlenote{Dr.~Paay insisted her name be first. :P}\\
       \affaddr{Research Centre for Socio-Interactive Design}\\
       \affaddr{Selma Lagerlofs Vej 300}\\
       \affaddr{Aalborg East 9220, Denmark}\\
       \email{Jeni@cs.aau.dk}
% 2nd. author
\alignauthor
Bjarke Martin Lauridsen\\
       \affaddr{Institute for Computer Science}\\
       \affaddr{Selma Lagerlofs Vej 300}\\
       \affaddr{Aalborg East 9220, Denmark}\\
       \email{blauri10@student.aau.dk}
% 3rd. author
\alignauthor
Elias Ringhauge\\
       \affaddr{Institute for Computer Science}\\
       \affaddr{Selma Lagerlofs Vej 300}\\
       \affaddr{Aalborg East 9220, Denmark}\\
       \email{eringh10@student.aau.dk}
\and  % use '\and' if you need 'another row' of author names
% 4th. author
\alignauthor
Eric Vignola Ruder\\
       \affaddr{Institute for Computer Science}\\
       \affaddr{Selma Lagerlofs Vej 300}\\
       \affaddr{Aalborg East 9220, Denmark}\\
       \email{eruder10@student.aau.dk}
% 5th. author
\alignauthor
Ivan Svilenov Penchev\\%\titlenote{This author is the one who did all the really hard work.}\\
       \affaddr{Institute for Computer Science}\\
       \affaddr{Selma Lagerlofs Vej 300}\\
       \affaddr{Aalborg East 9220, Denmark}\\
       \email{ipench14@student.aau.dk}
}
% 6th. author
%\alignauthor Charles Palmer\\
%       \affaddr{Palmer Research Laboratories}\\
%       \affaddr{8600 Datapoint Drive}\\
%       \affaddr{San Antonio, Texas 78229}\\
%       \email{cpalmer@prl.com}
%}
% There's nothing stopping you putting the seventh, eighth, etc.
% author on the opening page (as the 'third row') but we ask,
% for aesthetic reasons that you place these 'additional authors'
% in the \additional authors block, viz.

%\additionalauthors{Additional authors: John Smith (The Th{\o}rv{\"a}ld Group,
%email: {\texttt{jsmith@affiliation.org}}) and Julius P.~Kumquat
%(The Kumquat Consortium, email: {\texttt{jpkumquat@consortium.net}}).}
%\date{30 July 1999}
% Just remember to make sure that the TOTAL number of authors
% is the number that will appear on the first page PLUS the
% number that will appear in the \additionalauthors section.


\maketitle %Must be after autor declaration as it also makes that part

%%%%%%%%%%%%
%INPUTS
%%%%%%%%%%%%
% !TEX root = ../paper.tex
\begin{abstract}
\todo[inline]{Write abstract}

\todo{Add section numbers to each section because of references to certain sections of the paper}

\end{abstract}

\keywords{ACM proceedings; \LaTeX; text tagging}
% !TEX root = ../literature.tex
\section{Introduction}
Weiser\cite{Weiser:1991} points out:
 
{\em``Ubiquitous computing names the third wave in computing, just now beginning. First were mainframes, each shared by lots of people. Now we are in the personal computing era, person and machine staring uneasily at each other across the desktop. Next comes ubiquitous computing, or the age of calm technology, when technology recedes into the background of our lives.''}.

Weiser's vision of ubiquitous computing, is over 20 years old, includes the ubiquity availability of computers that are preferably not distinguishable from everyday objects. He also talks about "calm technology", which is this technology that resides in the periphery and plays a non-dominant role in a user's life. The perception of devices plays a major role in Weiser's vision, however the other side of that is availability of devices and also computing power, which is important for cross-device interaction. As so we could contend and regard that the latest incarnation of Weiser's vision is cross-device interaction, where ideally joining several devices would lead to single seamless, and natural user interaction, flexible, and not restricted to a few configurations\cite{Radle:2015}.
In order for this interaction to happen we need to remain as close as possible to the real world and have multimodality in mind, as Jain et al. states: ``human interaction with the world is multi-modal, and rich multi-modal interaction is part of what defines a natural experience.''\cite{Jain:2011}. 
This corresponds well witth the way Wigdor and Wixon defines NUI , where "Natural" is about how the users feel and what they do when using a product. The products must mirror the user's capabilities and meet their needs, but the trick is to help the users feel comfortable about the product without the need for much practice\cite{Wigdor:2011}. This becomes important for the design of product that has a non-dominant role in a user's life as the user should not be using a lot of time for adapting to the products.   
Ideally we would see a combination of multiple modalities, for instance gestures, augmented reality, touch, voice recognition etc. 
Researchers have made and published breakthroughs and designers constructed innovation design in each modality individually, however there has only been little research and work in combining them \cite{Jain:2011}. \\

It is possible for the existence of untapped opportunities in the integration of multiple natural user interaction modalities for enriching the cross-device experience with hand-held devices and large public displays. \\

In this paper we review the research that has been done within interaction with large displays. 
This has been done to map the current understanding and practices, thereby helping future researchers, within this field, to gain a perspective as well as identifying possible untapped opportunities for future work.
% !TEX root = ../literature.tex
\section{State of art}
We present an overview of research in interaction techniques for cross device interaction, with an emphasis on the natural user interface.
By looking at Public Space we found the following topics in HCI: Cross-device Interaction, Large Displays, Multi-device Interaction, Natural User Interface, Proxemics, Social Interaction, Tabletop Interaction, and Tangible Interaction. 
We used these topics for our inquiry in ACM DigitalLibrary, SpringerLink and Microsoft Research, which matched a total of 56 papers. 
To narrow down our focus we selected papers that matched a set of strict criteria: (1) The paper should contain one or more interaction techniques, (2) It is important that the techniques are evaluated using either interviews, surveys, field studies or any other research method.
The final set contained 34 literature sources which we further examined, by writing a short summary for each paper which included ``title, author's thesis (aim of paper), objectives (goals, summaries), methodologies used, findings, conclusion, and keywords''. 
Using applied thematic analysis different themes were induced on the summaries, which we later categorized into 5 themes: interaction with large displays, interaction with public displays, natural user interactions, cross-device interactions, cross-device natural user interactions.
The identified themes and the specific papers correlated to them are presented in Table 1, with full citations in the references section of this paper. 

\begin{table*}[t]
\centering
\begin{tabular}{@{}ll@{}}
\toprule
Theme & Paper \\ \midrule
Interacting with Large displays      &      [1] [3] [32][33] \\
Interacting with Public displays      &       [4] [5] [6] [7] [8] [10] [11] [12] [34] \\
Natural user interactions      &      [9] [13] [14] [15] [16] [17] [18] [19] [30] [31] \\
Cross-device interactions      &      [20] [21] [22] [23] [24] [25] \\
Cross-device natural user interactions      &      [26] [27] [28] [29] \\ \bottomrule
\end{tabular}
\caption{Among the final 34 papers found, we identified 5 themes.}
\label{table:themes}
\end{table*}
\section{Conclusion}\label{Sec:Conclusion}
In this paper we presented an overview of HCI research within interacting with large displays. We reviewed 34 papers from the last decade in detail. A picture (\Cref{fig:litreview}) was drawn about the relation of large displays, public displays, cross-device interaction and natural user interaction. Cross device interaction and natural user interaction was found to be a subset of large- and public displays. Based on further examination of these areas in research, we identified cross-device natural user interaction as well as some opportunities and shortcomings that we used to suggest possible future research areas.\\
In the future we would like to see quantitative research within the field of cross-device natural user interaction, as well as use of the produced statistically solid results as a guide for the design of further applications for cross-device interactions and large displays.

%Refferences and apendix are just after this weird block

%%%%%%%%%%%%%%%%%%%%%%%%%%%%%%%%%%%%%%%%%%%%%%%
% CCSXML IF needed
%
% The code below should be generated by the tool at
% http://dl.acm.org/ccs.cfm
% Please copy and paste the code instead of the example below. 
%
%\begin{CCSXML}
%<ccs2012>
% <concept>
%  <concept_id>10010520.10010553.10010562</concept_id>
%  <concept_desc>Computer systems organization~Embedded systems</concept_desc>
%  <concept_significance>500</concept_significance>
% </concept>
% <concept>
%  <concept_id>10010520.10010575.10010755</concept_id>
%  <concept_desc>Computer systems organization~Redundancy</concept_desc>
%  <concept_significance>300</concept_significance>
% </concept>
% <concept>
%  <concept_id>10010520.10010553.10010554</concept_id>
%  <concept_desc>Computer systems organization~Robotics</concept_desc>
%  <concept_significance>100</concept_significance>
% </concept>
% <concept>
%  <concept_id>10003033.10003083.10003095</concept_id>
%  <concept_desc>Networks~Network reliability</concept_desc>
%  <concept_significance>100</concept_significance>
% </concept>
%</ccs2012>  
%\end{CCSXML}

%\ccsdesc[500]{Computer systems organization~Embedded systems}
%\ccsdesc[300]{Computer systems organization~Redundancy}
%\ccsdesc{Computer systems organization~Robotics}
%\ccsdesc[100]{Networks~Network reliability}
%%%%%%%%%%%%%%%%%%%%%%%%%%%%%%%%%%%%%%%%%%%%%%%

%
%  Use this command to print the description
%
%\printccsdesc  (no idea what it is used  for)


%%%%%%%%%%%%
%REFFERENCE SECTION
%%%%%%%%%%%%
\printbibliography[heading=bibnumbered, title={References}]

%
% ACM needs 'a single self-contained file'!
%

%%%%%%%%%%%%%%%%%%%
%APPENDICES are optional
%\balancecolumns
%\appendix % THIS SETS A STYLE
%%Appendix A
%\section{Headings in Appendices}
%The rules about hierarchical headings discussed above for
%the body of the article are different in the appendices.
%In the \textbf{appendix} environment, the command
%\textbf{section} is used to
%indicate the start of each Appendix, with alphabetic order
%designation (i.e. the first is A, the second B, etc.) and
%a title (if you include one).  So, if you need
%hierarchical structure
%\textit{within} an Appendix, start with \textbf{subsection} as the
%highest level. Here is an outline of the body of this
%document in Appendix-appropriate form:
%\subsection{Introduction}
%\subsection{The Body of the Paper}
%\subsubsection{Type Changes and  Special Characters}
%\subsubsection{Math Equations}
%\paragraph{Inline (In-text) Equations}
%\paragraph{Display Equations}
%\subsubsection{Citations}
%\subsubsection{Tables}
%\subsubsection{Figures}
%\subsubsection{Theorem-like Constructs}
%\subsubsection*{A Caveat for the \TeX\ Expert}
%\subsection{Conclusions}
%\subsection{Acknowledgments}
%\subsection{Additional Authors}
%This section is inserted by \LaTeX; you do not insert it.
%You just add the names and information in the
%\texttt{{\char'134}additionalauthors} command at the start
%of the document.
%\subsection{References}
%Generated by bibtex from your ~.bib file.  Run latex,
%then bibtex, then latex twice (to resolve references)
%to create the ~.bbl file.  Insert that ~.bbl file into
%the .tex source file and comment out
%the command \texttt{{\char'134}thebibliography}.
%% This next section command marks the start of
%% Appendix B, and does not continue the present hierarchy
%\section{More Help for the Hardy}
%The sig-alternate.cls file itself is chock-full of succinct
%and helpful comments.  If you consider yourself a moderately
%experienced to expert user of \LaTeX, you may find reading
%it useful but please remember not to change it.
%%\balancecolumns % GM June 2007
%% That's all folks!

\end{document}