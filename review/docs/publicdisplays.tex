% !TEX root = ../literature.tex
\subsection{Interacting with Public Displays}

Rapid progression in the technology for displays and projections, led to considerable proliferation of large displays, thanks also to their capacity to promote activity and social awareness \cite{Huang:2003}, they moved out from research laboratories into public spaces. 
However, \emph{``these displays typically present predetermined feeds offering no interactivity''} \cite{Brignull:2003} states Brignull and Rogers. 
Ten years later Ojala et. al still support this thesis, that \emph{``to date, however, these displays are still used primarily as one-way commercial digital signs.''} \cite{Ojala:2012:MIP:2225044.2225065}. \\

Nevertheless researchers continue to combat and change this trend, as such a body of research has formed around interactive public displays. \\

A large part of this body presents unique  solutions for installations and devices for particular public settings with specific display technology \cite{Schieck:2012:AEM:2393132.2393141}.\\

Boring and Baur address the problem of crafting interaction techniques that can be used in an array of settings while at the same time maintain some autonomy from the components of public space and different display technology. 
They have done so by developing a \emph{``[...] conceptual framework and technical implementation that rely solely on the public displays and users' mobile devices.''} \cite{Boring:2013}. By  leveraging cell phone cameras they have enabled from-a-distance interaction with any public-display technology.\\

From-a-distance interaction using a handheld devices is not the only way to interact with public displays. 
Jacucci et al. adopted touch as a base for their ``Worlds of Information'' system. 
They \emph{``discuss how to design for and evaluate engagement in a public walk-up-and-use installation''} \cite{Jacucci:2010}. 
Also, low-cost motion detecting depth cameras permit gesture interactions with displays. 
Gang Ren and his team explain how they used input based on gestures to navigate 3D imagery and how such interaction techniques influence social dynamics around the display. 
They point out that\emph{ ``for large public displays, gestural interaction can enhance the experience of not only the current user but also the people sharing that public space or activity.''} \cite{Ren:2013}.\\

This observation led to value which could be exploited, as Lucero et. al. tried to do. 
Focusing on the collaborative and cross-device side, Lucero et. al. created MobiComics with the purpose to \emph{``explore shared collocated interaction with mobile phones and public displays in indoor public place.''} \cite{Lucero:2012}. 
Valkanova et. al tried to focus on the discussion and natural user interaction side by creating MyPossition, a \emph{``public display in the form of a large projection, featuring an interactive poll visualisation.''} \cite{Valkanova:2014} which aimed to aggregate public opinions for local issues by utilizing voting with gestures.\\

In this section we explored literature on interacting with public displays. 
By doing so, two clusters of research papers were formed, one on cross-device interaction, and another on natural user interaction, which defines the need for their further examination in relation to the main topic.