% !TEX root = ../literature.tex
\subsection{Discussion}
%this section needs to be called discussion, and you need to use it to discuss a bit more about the relationships between the different fields, concluding with the idea that some sharing of information across these areas to futher research various aspects would be a good idea. So you are on the right track - but you need to be a bit more detailed in your thinking
This paper aims to analytically summarize published research within the areas of interaction with large displays, natural user interaction, and cross device interaction.In this section we expand on the relationships between presented different fields, we believe this will further deepen our understanding of them.

%The findings of this work are that gesture recogntion using touchscreens is now a mature enough technology that it is ready for implementation in commerical products. The iPhone has shown that basic gesturing has a place in mobile operating systems and my results show that more complex gestures can be easily recognised and recognised accurately. The problem is context.
By making the objects that surround us change their non-interactive state to a responsive medium, Buxton\cite{Buxton:2000} and Wellner\cite{Wellner:1993} achieved a novelty supporting Weiser's vision of ubiquitous computing. A secondary objective of Weiser was related to the size of the object. An example of this principle can be related to the work of Czerwinski\cite{Czerwinski:2003} that shows how size can correlate to the performance in a work environment.

A new state is often related to not only a technologically progress but also conceptual ideas. A Technological example would be the development of the Kinect that provide new interaction possibilities\cite{Wilson:2010}, and a conceptual example is to use tangible objects as a medium for interacting with the virtual world\cite{Rekimoto:1997, Keefe:2001}.
This change provides new opportunities and challenges for developing interaction techniques in the interaction space of natural user interaction and cross-device interaction. These two identified areas differ in-between, however they also have common ground, we present this idea in \Cref{fig:litreview}.\\
\begin{figure}[h!]
\centering
\includegraphics[width=0.5\textwidth]{docs/research_areas.pdf}
\caption{Model of research areas derived from interaction with large displays}
\label{fig:litreview}
\end{figure}
We reflect upon how the shift from private to public environment correlates with the challenges in natural user interaction and cross-device interaction for interaction techniques.
While there isn't a specific formula covering the creation of an ideal interaction technique, there are some guidances. For example Brignull\cite{Brignull:2003} advises to make the transition between an onlooker to participant less socially awkward. Cheung\cite[Cheung:2014} shows that there are specific barriers for interacting and presents a diagram of each state when said challenges are encountered, he advises that those should be addressed in every case. However we notice that the majority of papers we examined do not explicitly focus on the environment, instead they present unique solutions, without beforehand taking into consideration the settings.
From the research done on public space and public display we are able to identify 3 elements that directly impacts the creation of interaction techniques: \emph{accessibility}, \emph{learn-ability}, and \emph{social acceptance}. To interact with a system it is necessary that the technique can be used without obstacles. For instance a large display can be interacted with by either touch or mid-air gestures, but mid-air gestures risks being interrupted by people who walks between the user and the sensor. It is also important that the user understands how to interact with the system and \emph{``move beyond ephemeral interactions, driven by the playfulness of the interface''}\cite{Jacucci:2010}, especially for a public walk-up-and-use system. However in the end for the user to engage in using the system, the interaction technique must be acceptable in a social circumstances to avoid intimidating or embarrassing the user. 

The fields of natural user interactions and cross-device interactions were examined. 
We were surprised to see that research in combining features from the two areas is slim even though they are complementary to each other. 
There was research that was done \cite{Scharf:2013,Seifert:2012,Valkanova:2014,Vogel:2005} . 
Interaction-techniques were evaluated with 3-6 people using qualitative surveys in \cite{Seifert:2012,Valkanova:2014,Vogel:2005} and without qualitative surveys in \cite{Schmidt:2012}.\\

In summary, we can see research opportunities in further exploration of cross-device natural user interactions technique for large displays. We believe the limitations that each of these fields ignore is the interaction space between them. To move beyond this limitation, we believe a unification of these discrete interaction techniques in continuous interaction space and it's further research would be positive and beneficial.
Firstly, as handheld devices become thinner and lighter and spatially-aware technologies, such as Kinect, continue to evolve, this opens a potentially rich space of aforementioned techniques. 
Secondly, we see opportunities in quantitative aspect, for example comparative study of different interaction techniques. 
