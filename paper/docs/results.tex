% !TEX root = ../paper.tex
%http://sphweb.bumc.bu.edu/otlt/MPH-Modules/BS/BS704_HypothesisTesting-ANOVA/BS704_HypothesisTesting-Anova_print.html
\section{Results}
We will now present the results that we achieved through out our experiment and also how they were achieved. We will first present our findings in respect to time, then in respect to accuracy. We will then show that there is a significant interaction between our two independent variables, grid size and technique, with regards to success rate. Finally, we will look at the questionnaires and show the significant findings there. We will discuss these results later, in the \nameref{sec:discussion} section.  

\subsection{Target Hit Time}

Each interaction technique was completed 18 times per participant. In total, each technique was performed 954 times. \Cref{tab:meanTimesTechnique} shows the mean time per. target for each of the techniques as well as their standard deviation. 
\begin{table}[H]
	\centering
	\begin{tabular}{|c|c|c|c|c|}
		\hline
		\rowcolor[HTML]{9B9B9B} 
		 & \textbf{Pinch} & \textbf{Swipe} & \textbf{Throw} & \textbf{Tilt} \\ \hline
		Mean & 8.66  sec         & 5.71 sec          & 7.07 sec          & 6.00 sec         \\ \hline
		Std. Dev. & 6.56 sec & 3.76 sec & 6.08 sec & 3.89 sec \\ \hline
	\end{tabular}
	\caption{Mean time and standard deviation for each technique per. target.}
	\label{tab:meanTimesTechnique}
\end{table}

A one way, repeated measure ANOVA revealed, with a Greenhouse-Geisser correction (0.874), that the difference in time between the different techniques is in fact a significant($F(2.622, 2498.37)=67.192$, $p<0.001$). A post hoc pair-wise test using a LSD correction revealed that \pinch and \swipe($p<0.001$), \pinch and \tilt ($p<0.001$), \pinch and \throw ($p<0.001$), \swipe and \throw ($p<0.001$), and \throw and \tilt ($p<0.001$) were all significantly different from each other. However, \swipe and \tilt were not ($p=0.092$)

We then examined the effect of grid size on time. Each participant had a equal mixture of small and large grid targets. This means that each technique had 9 targets of each size. Every test had 36 targets of each size, totaling 1908 targets of each size, through out all tests. \Cref{tab:meanTimesSize} shows the mean time and standard deviation for each grid size. 

\begin{table}[H]
	\centering
	\begin{tabular}{|c|c|c|}
		\hline
		\rowcolor[HTML]{9B9B9B} 
		 & \textbf{Large Grid} & \textbf{Small Grid} \\ \hline
		Mean & 6.21 sec & 7.51 sec \\ \hline
		Std. Dev. & 4.88 sec & 5.71 sec \\ \hline
	\end{tabular}
	\caption{Mean time and standard deviation for each grid size per. target}
	\label{tab:meanTimesSize}
\end{table}

We applied another one way, repeated measures ANOVA($F(1,1907) = 77.587$, $p<0.001$). This means that the grid size has a significant effect on time when hitting the target. 

\subsection{Accuracy}

As above, we will perform an analysis on the effect of each technique on accuracy as well as the effect of each grid size. There are the same amount of data points as the above analysis. 

The mean and standard deviation for each technique can be seen in \Cref{tab:meanHitTechnique}. 

\begin{table}[H]
	\centering
	\begin{tabular}{|c|c|c|c|c|}
			\hline
			\rowcolor[HTML]{9B9B9B} 
			& \textbf{Pinch} & \textbf{Swipe} & \textbf{Throw} & \textbf{Tilt} \\ \hline
			Mean & 0.72         & 0.94          & 0.88          & 0.68         \\ \hline
			Std. Dev. & 0.45 & 0.25 & 0.32 & 0.47 \\ \hline
	\end{tabular}
	\caption{Mean hit  and standard deviation for each technique per. target.}
	\label{tab:meanHitTechnique}
\end{table}

We performed another one way, repeated measure ANOVA ($F(2.645,2420.246)=105.535$, $p<0.001$), which told us that the different techniques also had a significant effect on the hit ratio. We perform another pair-wise post hoc LSD test to see where the significant differences are. \pinch and \tilt had $p=0.094$, which means that there was no significant difference between these two techniques. All other techniques were significantly different from each other ($p<0.001$). 

We then looked at the grid sizes effect on accuracy. The mean and standard deviation can be seen in \Cref{tab:meanHitSize}. We performed another one-way, repeated measure ANOVA ($F(1,1907)=103.443$, $p<0.001$). This means that there is a significant effect from the grid sizes on the hit rate of each attempt.

\begin{table}[H]
	\centering
	\begin{tabular}{|c|c|c|}
		\hline
		\rowcolor[HTML]{9B9B9B} 
		& \textbf{Large Grid} & \textbf{Small Grid} \\ \hline
		Mean & 0.87 & 0.74 \\ \hline
		Std. Dev. & 0.34 & 0.44 \\ \hline
		\end{tabular}
		\caption{Mean hit and standard deviation for each grid size per. target}
		\label{tab:meanHitSize}
\end{table}

\subsection{Interaction between Grid Size and Technique}
We also wanted to see if there was any significant interaction between our two main independent variables, grid size and technique. 
To see if the interaction between grid size and technique was having any effect on time, we performed a two way, repeated measures ANOVA with a Greenhouse-geisser correction(0.834)($F(2.503,1191.556)=0.114$, $p=0.906$). This showed no significant interaction. We then performed the same test, this time trying to see if the interaction had any effect on the hit rate(Greenhouse-geisser correction(0.896), $F(2.688,1279.720)=5.741$, $p=0.001$). This shows that the interaction between grid size and technique does in fact have a significant effect on success rate of the attempt. 

When then split the data up between the two grid sizes. We then performed a one way, repeated measure ANOVA, with the techniques as an effect and a Greenhouse-geisser correction(0.824), on all the large grid targets ($F(2.472, 1176.749)=42.773$, $p<0.001$).
Performing a LSD pair-wise post hoc test shows that \pinch and \tilt ($p=1$) are not significantly different. \pinch and \swipe ($p<0.001$), \pinch and \throw ($p<0.001$), \swipe and \throw ($p=0.024$),  \swipe and \tilt ($p<0.001$) and \tilt and \throw ($p<0.001$) were all significantly different from each other.

We then repeated the same test on all small grid targets (Greenhouse-geisser correction (0.907), $F(2.722, 1295.674)=62.754$, $p<0.001$). A post hoc, LSD pair-wise comparison showed that \pinch and \swipe ($p<0.001$), \pinch and \throw ($p<0.001$), \pinch and \tilt ($p=0.031$), \swipe and \throw ($p=0.001$), \swipe and \tilt ($p<0.001$) and \throw and \tilt ($p<0.001$) were all significantly different. 