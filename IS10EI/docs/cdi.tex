% !TEX root = ../literature.tex
\subsection{Cross-Device Interaction}
Interaction was understood, usually, as single user using a single device, this was almost always a desktop computer\cite{Levin:2014}. We have moved from this protocol into the more complex world of cross-device interaction, as such it is important to give definition of this phenomenon.

\emph{"Cross-device interaction is the type of interaction, where human users interact with multiple separate input and output devices, where input devices will be used to manipulate content on output devices within a perceived interaction space with immediate and explicit feedback."} \cite{Scharf:2013}.

Rikimoto presents an early envisionment of this idea. In his work the technique he proposed was a Prick-and-Drop technique, he explains that "Pick-and-Drop allows a user to pick up an object on a display and drop it on another display as if he/she were manipulating a physical object." \cite{Rekimoto:1997}. The author discusses how, even though from a functional point of view, the technique does not differ from a traditional remote copy command, the added extra layer of physicalness makes a difference for the user. \emph{"understanding that we are living in a fusion of physical (real) and virtual (computer) worlds. Each has its own advantages and disadvantages. Pick-and-Drop, for example, adds physicalness to user interfaces, because we feel that traditional data transfer methods are too virtual and hard to learn due to their lack of physical aspects"} \cite{Rekimoto:1997} remarks the writer.

Perhaps one of the earliest working systems illustrating CDI is by Myers et al.
\cite{Myers:2001} one of the application realized within their \emph{Pebbles project} is SlideShow Commanded that utilized Personal Devices Assistants (PDA) to control a PowerPoint presentation running on other computer or laptop.
It was possible not just moving between slides, but also scribbling and writing on the PDA slides, while annotations are shown on the presentation for the audience.

A more modern reading on this comes from Hamilton and Wigdor, they wanted to see how the users would utilize a group of tablets with cross-device capabilities. For this task they crafted\emph{ "Conductor, a prototype system that combines a set of interaction techniques designed to enable single user, cross-device applications, manage cross-device relationships, and enable easy functionality chaining."} \cite{Hamilton:2014} to achieve this task they uses traditional menus with color-coded device names to select the tablets to share information with or to chain tasks across them. However the experiment and the user study done affirmed that keeping track of multiple, often very similar devices can represent a surprisingly significant challenge, and also that users extensively used spatial configurations of tablets for categorical organization. In other words, although menus are a familiar and expected way to select items which have no clear spatial relation, they might seem burdensome for the purpose of selecting one out of many devices from a spatial configuration.  

This spatial aspect makes Radle et. al continued with the multi-tablet cross-device interaction, and ask what is the appropriate design of this cross-device interaction. In other words their focus shift to the competing interaction approaches such as spatially-aware vs. spatially-agnostic techniques. In order to do that they firstly define three main categories of cross-device interactions: 
\begin{enumerate}[topsep=0pt,itemsep=1ex,partopsep=1ex,parsep=1ex]
	\item Synchronous gestures
	\item Spatially-agnostic interactions
	\item Spatially-aware interactions. 
\end{enumerate} After which they perform a two-phase user study that explores the design space. Based on the experiments they found that \emph{"the results indicate that spatially-aware techniques are preferred by users and can decrease mental demand, effort, and frustration, but only when they are designed with great care."} \cite{Radle:2015}.

Cross interaction between the same type of device are not the only one. Schmidt et al. propose a cross-device interaction style for mobiles and surfaces. The researchers point out that \emph{"natural forms of interaction have evolved for personal devices that we carry with us (mobiles) as well as for shared interactive displays around us (surfaces) but interaction across the two remains cumbersome in practice"} \cite{Schmidt:2012}; so in order to overcome this they propose to use mobiles as tangible input on the surface in a stylus like fashion. While Schmidt et al. focuses on close interactions Boring et. al. tackles the in-distance aspect. More specifically they explored
the transfer of data from a large public display onto a mobile
device by using the camera on said device.  They show
that there is a need for enabling data exchange between mobile
devices and public displays. Their idea is to combine the large-scale visual output and the mobile phone as an input, to enable a channel of communication, which is bidirectional, with the large public display. To do this they \emph{"propose and compare three different interaction techniques (Scroll, Tilt and Move) for continuous control of a pointer located on a remote display using a mobile phone"} \cite{Boring:2009}.

Skov et al. \cite{Skov:2015} illustrate six different cross-device interaction techniques for the case of card playing in \emph{Investigating Cross-Device Interaction techniques}.
A player can see their own cards on their phone and use three different techniques for playing a card from the handheld to the tablet, which is placed on a table.
In the other direction, i.e. when drawing a card, the player also has three techniques to choose from.
The usability study aims to quantitatively evaluate each of the techniques and showed that there is a difference in time and number of errors between the techniques. The findings that the authors present show that coping the natural gesture of card playing in the form of wrist-whip interaction technique, resulted in a slower than the other techniques, which the users reported as less useful and natural. In regards to interaction errors the swipe interaction technique causes significant problems which results in participants making more errors. The field study presented by the authors shows that when playing in real card game situations, people mix up the techniques and thus perform a wrong action.

The current paragraph exemplifies that designing a solution for cross-device interaction can be complex, as such to understand  unique properties and challenges of the domain is a must. The early literature focuses on interaction support. Gradually this evolves to understanding the interaction process itself in different device environment i.e. large display and hand-held device, and multiple hand-held device.With the body of knowledge on cross-device, and natural interactions as part of large display, a logical step is to look and ask if researchers have tried to combine CDI and NUI in public spaces with large displays.
