% !TEX root = rapport.tex

% Giver mulighed for at styre captions for figurer, tabeller og lignende
%\usepackage[justification=centerlast, font=it,labelfont=bf]{caption}

% Udskriver side antal med \pageref{LastPage}
\usepackage{lastpage}
%\usepackage{longtable}

% Table of Contents opsætning
%\usepackage{tocloft}
%\setlength{\cftpartnumwidth}{1.2cm}

%Til tabular positioning
%\usepackage{float}
%\restylefloat{table}

%Theoms and definitions
%\usepackage{amsthm}
%\newtheorem{mydef}{Definition}

%\usepackage{booktabs}

% Pakke så vi kan bruge farver
%\usepackage[table]{xcolor} %Same package, but working on tables

%more colors
%\usepackage{color}

% Todo notes - [disable] slår todo fra
%\usepackage[disable]{todonotes}
%\usepackage{todonotes}
%\setlength{\marginparwidth}{2,2cm}

% Dansk sprog
%\usepackage[english]{babel}

% Retter problemet med at skriften ikke vises korrekt - source: http://dsanta.users.ch/resources/type1.html
%\usepackage{ae,aecompl}

% Font indstillinger
% Hjælper med orddeling ved æ, ø og å. Sætter fontene til at være ps-fonte, i stedet for bmp
%\usepackage[utf8]{inputenc}% Brug af æ, ø og å
%\usepackage[T1]{fontenc}
%\usepackage{textcomp}
%\usepackage{fourier}

% Pakke til billeder
%\usepackage{graphicx}

% Pakker til sub-billeder (side-by-side)
%\usepackage{caption}
%\usepackage{subcaption}


% Pakke til at kunne sætte tekst op ved siden af billeder
%\usepackage{wrapfig}

% Pakke til rammer
%\usepackage{framed}

% Pakker til at vise kildekode
%\usepackage{color}
%\usepackage{listings}
%% !TEX root = rapport.tex

\lstdefinelanguage{CSharp}
{
 morecomment = [l]{//}, 
 morecomment = [l]{///},
 morecomment = [s]{/*}{*/},
 morestring=[b]", 
 morestring=[b]', 
 sensitive = true,
 morekeywords = [1]
 {
   abstract,  event,  new,  struct, var,
   as,  explicit,  null,  switch,
   base,  extern,  object,  this,
   bool,  false,  operator,  throw,
   break,  finally,  out,  true,
   byte,  fixed,  override,  try,
   case,  float,  params,  typeof,
   catch,  for,  private,  uint,
   char,  foreach,  protected,  ulong,
   checked,  goto,  public,  unchecked,
   class,  if,  readonly,  unsafe,
   const,  implicit,  ref,  ushort,
   continue,  in,  return,  using,
   decimal,  int,  sbyte,  virtual,
   default,  interface,  sealed,  volatile,
   delegate,  internal,  short,  void,
   do,  is,  sizeof,  while,
   double,  lock,  stackalloc,   
   else,  long,  static, yield,
   enum,  namespace,  string
   },
 morekeywords = [2]
 {
  String, Color, List, KeyPressEventArgs, KeyEventArgs, ControllerArgs 
  Console, DateTime, Form,
  Homepage, Login, EditProblem, ReportProblem, ProblemList, ProblemControl, VandalismControl, RoadControl, SentReports, ProblemDetails,
  Homepageform, LoginForm, NewReport, ReportList, ExistingUserControl, NewUserControl, ProblemControl, RoadControl, VandalismControl,
  Problem, ProblemCollection, ProblemTypeRoad, ProblemTypeVandalism, Resident, Worker, Images, ProblemHistory, WorkerComment, Area,
  DataManager, NeighbourhoodwatchEntities, EntityObject, EntityState,
  PointLatLng, DensityPoint, DensityCluster, DensityAlgorithm,
  WorkerType, Department, PD_Severity, PD_Severety, PD_Type, PD_EnvType, PD_RoadUsability, PD_RoadMaterial, PD_RoadType, PD_VandalismType,
  },
%morekeywords = [3]
%{
%abstract, as, base, break, case,
%catch, checked, class, const, continue,
%default, delegate, do, else, enum,
%event, explicit, extern, false,
%finally, fixed, for, foreach, goto, if,
%implicit, in, interface, internal, is,
%lock, namespace, new, null, operator,
%out, override, params, private,
%protected, public, readonly, ref,
%return, sealed, sizeof, stackalloc,
%static, struct, switch, this, throw,
%true, try, typeof, unchecked, unsafe,
%using, virtual, volatile, while, bool,
%byte, char, decimal, double, float,
%int, lock, object, sbyte, short, string,
%uint, ulong, ushort, void,
%from, where, select, group, into, orderby,
%join, let, in, on, equals, by,
%ascending, descending},
%morecomment=[l]{//},
%morecomment=[s]{/*}{*/},
%morecomment=[l][keywordstyle4]{\#},
%morestring=[b]",
%morestring=[b]',
%}
}

\lstset{ % Sets default values for the listing environments
	backgroundcolor=\color{black!3!},
	tabsize=2,
	basicstyle=\footnotesize,%\scriptsize,
	escapechar=�,
	columns=fixed,
	showstringspaces=false,
	extendedchars=true,
	breaklines=true,
	breakatwhitespace=true,
	prebreak = \raisebox{0ex}[0ex][0ex]{\ensuremath{\hookleftarrow}},
	frame=tb,
	framextopmargin=6pt,
	framexbottommargin=6pt,
%	frame=lrb, 
	xleftmargin=\fboxsep, xrightmargin=-\fboxsep,
	rulecolor=\color{black!15!},
	showtabs=false,
	showspaces=false,
	showstringspaces=false,
	identifierstyle=\ttfamily,
	keywordstyle=\ttfamily \color[rgb]{0,0,1}, % Standard keywords
%	keywordstyle=[3]\ttfamily \color[rgb]{0,0,1}, % More standard keywords
	keywordstyle=[2]\ttfamily \color[rgb]{0.17,0.57,0.69}, % My keywords
	commentstyle=\ttfamily \color[rgb]{0.133,0.545,0.133},
	stringstyle=\ttfamily \color[rgb]{0.64,0.08,0.08},
	numbers=left, numberstyle=\tiny, numbersep=5pt,
	escapeinside={@@}{@@},
	language=CSharp
}

% Indstillinger til caption
\DeclareCaptionFont{white}{\color{black}}
\DeclareCaptionFormat{listing}{%
  \parbox{\textwidth}{\colorbox{black!10!}{\parbox{\textwidth}{\hspace{8pt}#1#2#3}}}}
\captionsetup[lstlisting]{format=listing, labelfont=bf, textfont=white, singlelinecheck=false, margin=0pt, skip=6pt}

\lstnewenvironment{code}[1][]%
  {\minipage{\linewidth} 
   \lstset{basicstyle=\ttfamily\scriptsize,frame=single,#1}}
  {\endminipage}

\makeatletter
\lst@AddToHook{TextStyle}{\let\lst@basicstyle\ttfamily\normalsize\fontfamily{pcr}\selectfont}
\makeatother

%\usepackage{amsmath}
%\usepackage{amssymb}

%\usepackage{rotating}


%\usepackage{multirow}

%\usepackage[bookmarksdepth=3, hidelinks]{hyperref} % http://ctan.org/pkg/hyperref
%\usepackage{bookmark} % http://ctan.org/pkg/bookmark

%\usepackage{cleveref}



