% !TEX root = ../literature.tex
\section{Introduction}
In historic relation, most applications are found on devices that are used in personal context like a laptop, a personal computer, a mobile phone and so forth. However as our, as a race, technical capabilities grew, so did our dreams, and we started imagining new places and uses, for the current and future technology, in order to make our lives easier, more entertaining or even just different. 

An example of this statement can be seen in replacement of the traditional signs with digital displays. This is due to rapid progression in the technology for displays and projections, which naturally led to lowering of the marginal price, thus \emph{``displays have moved out of research laboratories into public spaces ''} \cite{Hinrichs:2013:IPD:2478559.2478965}. Now we have displays that are deployed in malls, shop windows and even in urban environments.We can see the transformation from traditional to digital in Inner city areas, airports, train stations, and stadiums. And with it we can enable a new forms of multimedia presentation and new user experiences. However it is an open question do we use this opportunity, as the vast majority of digital displays, for the last 10 years, remain non interactive.

This trend, that is happening, of broadening the domain of multimedia to move beyond the private, into the public space is promising, as it not only gives the possibility for technology to become more relevant for our lives, but it also presents possible more immersive experience. Presented with this opportunity research has been done in the area of digital displays for use in  public context, marked with the term public displays. Some of them are a specific system for a specific task, while others have worked on abstract level for example Bringhul et al.\cite{Brignull:2003} identified and tried to tackle the problem about how to entice people to participate in public displays, Muller et al.\cite{Muller:2010} analysis the public design space contributing to the understanding of the mental and interaction models. Cheung et al. \cite{Cheung:2014} goes further to analyses the barriers you face when interacting with a public displays with a mobile device.  What all this authors have in common, beside analyzing the challenges for public displays, is that they agree that a key component of a public display is the interaction, marked with the term public interactive display.

There are different ways to interact with a public display. One way a person can do so is via a second device (cross-device interaction) or via gaze, touch or some other natural modality (natural user interactions), however recent research in HCI has focused on combining those two fields in what should be known as cross-device natural user interaction. An example can be seen within the work of some researchers used spatial information \cite{Marquardt:2011, Marquardt:2012}, others used touch \cite{Seifert:2012}, or combined touch with air gestures \cite{Bragton:2011}

In this paper we review the research that has been done within interaction with large displays. 
This has been done to map the current understanding and practices, thereby helping future researchers, within this field, to gain a perspective as well as identifying possible untapped opportunities for future work.