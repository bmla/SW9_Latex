% !TEX root = rapport.tex

% Giver mulighed for at styre captions for figurer, tabeller og lignende
%\usepackage[justification=centerlast, font=it,labelfont=bf]{caption}

% Udskriver side antal med \pageref{LastPage}
\usepackage{lastpage}
%\usepackage{longtable}

% Table of Contents opsætning
%\usepackage{tocloft}
%\setlength{\cftpartnumwidth}{1.2cm}

%Til tabular positioning
%\usepackage{float}
%\restylefloat{table}

%Theoms and definitions
%\usepackage{amsthm}
%\newtheorem{mydef}{Definition}

%\usepackage{booktabs}

% Pakke så vi kan bruge farver
%\usepackage[table]{xcolor} %Same package, but working on tables

%more colors
%\usepackage{color}

% Todo notes - [disable] slår todo fra
%\usepackage[disable]{todonotes}
%\usepackage{todonotes}
%\setlength{\marginparwidth}{2,2cm}

% Dansk sprog
%\usepackage[english]{babel}

% Retter problemet med at skriften ikke vises korrekt - source: http://dsanta.users.ch/resources/type1.html
%\usepackage{ae,aecompl}

% Font indstillinger
% Hjælper med orddeling ved æ, ø og å. Sætter fontene til at være ps-fonte, i stedet for bmp
%\usepackage[utf8]{inputenc}% Brug af æ, ø og å
%\usepackage[T1]{fontenc}
%\usepackage{textcomp}
%\usepackage{fourier}

% Pakke til billeder
%\usepackage{graphicx}

% Pakker til sub-billeder (side-by-side)
%\usepackage{caption}
%\usepackage{subcaption}


% Pakke til at kunne sætte tekst op ved siden af billeder
%\usepackage{wrapfig}

% Pakke til rammer
%\usepackage{framed}

% Pakker til at vise kildekode
%\usepackage{color}
%\usepackage{listings}
%\input{csharp}

%\usepackage{amsmath}
%\usepackage{amssymb}

%\usepackage{rotating}


%\usepackage{multirow}

%\usepackage[bookmarksdepth=3, hidelinks]{hyperref} % http://ctan.org/pkg/hyperref
%\usepackage{bookmark} % http://ctan.org/pkg/bookmark

%\usepackage{cleveref}



