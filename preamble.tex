% !TEX root = rapport.tex

\documentclass[a4paper,12pt, twoside, openright]{memoir}

% Giver mulighed for at styre captions for figurer, tabeller og lignende
\usepackage[justification=centerlast, font=it,labelfont=bf]{caption}

% Udskriver side antal med \pageref{LastPage}
\usepackage{lastpage}
\usepackage{longtable}

% Table of Contents opsætning
\usepackage{tocloft}
\setlength{\cftpartnumwidth}{1.2cm}

%Til tabular positioning
\usepackage{float}
%\restylefloat{table}

%Theoms and definitions
\usepackage{amsthm}
\newtheorem{mydef}{Definition}

\usepackage{booktabs}

% Pakke så vi kan bruge farver
\usepackage[table]{xcolor} %Same package, but working on tables

%more colors
\usepackage{color}

% Todo notes - [disable] slår todo fra
%\usepackage[disable]{todonotes}
\usepackage{todonotes}
%\setlength{\marginparwidth}{2,2cm}

% Dansk sprog
\usepackage[english]{babel}

% Retter problemet med at skriften ikke vises korrekt - source: http://dsanta.users.ch/resources/type1.html
\usepackage{ae,aecompl}

% Font indstillinger
% Hjælper med orddeling ved æ, ø og å. Sætter fontene til at være ps-fonte, i stedet for bmp
\usepackage[utf8]{inputenc}% Brug af æ, ø og å
\usepackage[T1]{fontenc}
\usepackage{textcomp}
\usepackage{fourier}

% Pakke til billeder
\usepackage{graphicx}

% Pakker til sub-billeder (side-by-side)
%\usepackage{caption}
%\usepackage{subcaption}


% Pakke til at kunne sætte tekst op ved siden af billeder
\usepackage{wrapfig}

% Pakke til rammer
\usepackage{framed}

% Pakker til at vise kildekode
\usepackage{color}
\usepackage{listings}
% !TEX root = rapport.tex

\lstdefinelanguage{CSharp}
{
 morecomment = [l]{//}, 
 morecomment = [l]{///},
 morecomment = [s]{/*}{*/},
 morestring=[b]", 
 morestring=[b]', 
 sensitive = true,
 morekeywords = [1]
 {
   abstract,  event,  new,  struct, var,
   as,  explicit,  null,  switch,
   base,  extern,  object,  this,
   bool,  false,  operator,  throw,
   break,  finally,  out,  true,
   byte,  fixed,  override,  try,
   case,  float,  params,  typeof,
   catch,  for,  private,  uint,
   char,  foreach,  protected,  ulong,
   checked,  goto,  public,  unchecked,
   class,  if,  readonly,  unsafe,
   const,  implicit,  ref,  ushort,
   continue,  in,  return,  using,
   decimal,  int,  sbyte,  virtual,
   default,  interface,  sealed,  volatile,
   delegate,  internal,  short,  void,
   do,  is,  sizeof,  while,
   double,  lock,  stackalloc,   
   else,  long,  static, yield,
   enum,  namespace,  string
   },
 morekeywords = [2]
 {
  String, Color, List, KeyPressEventArgs, KeyEventArgs, ControllerArgs 
  Console, DateTime, Form,
  Homepage, Login, EditProblem, ReportProblem, ProblemList, ProblemControl, VandalismControl, RoadControl, SentReports, ProblemDetails,
  Homepageform, LoginForm, NewReport, ReportList, ExistingUserControl, NewUserControl, ProblemControl, RoadControl, VandalismControl,
  Problem, ProblemCollection, ProblemTypeRoad, ProblemTypeVandalism, Resident, Worker, Images, ProblemHistory, WorkerComment, Area,
  DataManager, NeighbourhoodwatchEntities, EntityObject, EntityState,
  PointLatLng, DensityPoint, DensityCluster, DensityAlgorithm,
  WorkerType, Department, PD_Severity, PD_Severety, PD_Type, PD_EnvType, PD_RoadUsability, PD_RoadMaterial, PD_RoadType, PD_VandalismType,
  },
%morekeywords = [3]
%{
%abstract, as, base, break, case,
%catch, checked, class, const, continue,
%default, delegate, do, else, enum,
%event, explicit, extern, false,
%finally, fixed, for, foreach, goto, if,
%implicit, in, interface, internal, is,
%lock, namespace, new, null, operator,
%out, override, params, private,
%protected, public, readonly, ref,
%return, sealed, sizeof, stackalloc,
%static, struct, switch, this, throw,
%true, try, typeof, unchecked, unsafe,
%using, virtual, volatile, while, bool,
%byte, char, decimal, double, float,
%int, lock, object, sbyte, short, string,
%uint, ulong, ushort, void,
%from, where, select, group, into, orderby,
%join, let, in, on, equals, by,
%ascending, descending},
%morecomment=[l]{//},
%morecomment=[s]{/*}{*/},
%morecomment=[l][keywordstyle4]{\#},
%morestring=[b]",
%morestring=[b]',
%}
}

\lstset{ % Sets default values for the listing environments
	backgroundcolor=\color{black!3!},
	tabsize=2,
	basicstyle=\footnotesize,%\scriptsize,
	escapechar=�,
	columns=fixed,
	showstringspaces=false,
	extendedchars=true,
	breaklines=true,
	breakatwhitespace=true,
	prebreak = \raisebox{0ex}[0ex][0ex]{\ensuremath{\hookleftarrow}},
	frame=tb,
	framextopmargin=6pt,
	framexbottommargin=6pt,
%	frame=lrb, 
	xleftmargin=\fboxsep, xrightmargin=-\fboxsep,
	rulecolor=\color{black!15!},
	showtabs=false,
	showspaces=false,
	showstringspaces=false,
	identifierstyle=\ttfamily,
	keywordstyle=\ttfamily \color[rgb]{0,0,1}, % Standard keywords
%	keywordstyle=[3]\ttfamily \color[rgb]{0,0,1}, % More standard keywords
	keywordstyle=[2]\ttfamily \color[rgb]{0.17,0.57,0.69}, % My keywords
	commentstyle=\ttfamily \color[rgb]{0.133,0.545,0.133},
	stringstyle=\ttfamily \color[rgb]{0.64,0.08,0.08},
	numbers=left, numberstyle=\tiny, numbersep=5pt,
	escapeinside={@@}{@@},
	language=CSharp
}

% Indstillinger til caption
\DeclareCaptionFont{white}{\color{black}}
\DeclareCaptionFormat{listing}{%
  \parbox{\textwidth}{\colorbox{black!10!}{\parbox{\textwidth}{\hspace{8pt}#1#2#3}}}}
\captionsetup[lstlisting]{format=listing, labelfont=bf, textfont=white, singlelinecheck=false, margin=0pt, skip=6pt}

\lstnewenvironment{code}[1][]%
  {\minipage{\linewidth} 
   \lstset{basicstyle=\ttfamily\scriptsize,frame=single,#1}}
  {\endminipage}

\makeatletter
\lst@AddToHook{TextStyle}{\let\lst@basicstyle\ttfamily\normalsize\fontfamily{pcr}\selectfont}
\makeatother

% Pakker til matematik
\usepackage{amsmath}
\usepackage{amssymb}

% Pakke til landskab visning
%\usepackage{lscape}

% Til rotation af tekst
\usepackage{rotating}

% Til advancerede tabeller
\usepackage{multirow}

%%%%%%%%%%%%%%%%%%%%%%%%%%%%%%%%%%%%%%%%%
%	TILFØJ IKKE PAKKER HERUNDER - NEDENSTÅENDE SKAL LOADES SIDST!	   %
%%%%%%%%%%%%%%%%%%%%%%%%%%%%%%%%%%%%%%%%%

% Sætter [hyphens]-option når hyperref kalder url pakken. Dette gøres for korrekt at kunne  breake lange links.
\PassOptionsToPackage{hyphens}{url}

% PDF links og bookmarks (\usepackage{bookmark} fixer bookmark strukturen)
\usepackage[bookmarksdepth=3, hidelinks]{hyperref} % http://ctan.org/pkg/hyperref
\usepackage{bookmark} % http://ctan.org/pkg/bookmark

% Clever references - ja, det er mega smarte (automatiske henvisninger)
% Se her hvordan det benyttes: http://www.howtotex.com/packages/automatic-clever-references-with-cleveref
\usepackage{cleveref}
\crefname{paragraph}{\S}{\S\S} % default is {paragraph}{paragraphs}

% Padding mellem tabel-rækker
\renewcommand{\arraystretch}{1.4}

% Margin mellem items i itemize
\usepackage{enumitem}
\setlist{noitemsep,topsep=10pt,parsep=2pt,partopsep=2pt}

% Kommando der skal bruges hvis vi ønsker en caption OVER en figur - skal indsættes under centrerings-kommandoen
% Eks. \customtopcaption{Denne figur viser...}
\newcommand{\customtopcaption}[1]
{
	\textit{#1\medskip}
}

% TODO GROUP
\newcommand{\grouptodo}[2]
{
     \stepcounter{grouptodocounter}\todo[size=\footnotesize, color=LimeGreen!35]{\textbf{\colorbox{Black}{\color{White}\thegrouptodocounter: #1:} } #2}
}

% TODO STEFAN
\newcommand{\stefan}[2][ ]
{
  \ifthenelse{\equal{#1}{inline}}
    {\stepcounter{todocounter}\todo[inline, size=\footnotesize, color=NavyBlue!35]{\textbf{\color{NavyBlue}{\thetodocounter: Stefan:} } #2}}
    {\stepcounter{todocounter}\todo[size=\footnotesize, color=NavyBlue!35]{\textbf{\color{NavyBlue}{\thetodocounter: Stefan:} } #2}}
}

% TODO KAYSEN
\newcommand{\kaysen}[2][ ]
{
  \ifthenelse{\equal{#1}{inline}}
    {\stepcounter{todocounter}\todo[inline, size=\footnotesize, color=Green!35]{\textbf{\color{Green}{\thetodocounter: Kaysen:} } #2}}
    {\stepcounter{todocounter}\todo[size=\footnotesize, color=Green!35]{\textbf{\color{Green}{\thetodocounter: Kaysen:} } #2}}
}

% TODO HIEU
\newcommand{\hieu}[2][ ]
{
  \ifthenelse{\equal{#1}{inline}}
    {\stepcounter{todocounter}\todo[inline, size=\footnotesize, color=Orange!35]{\textbf{\color{Orange}{\thetodocounter: Hieu:} } #2}}
    {\stepcounter{todocounter}\todo[size=\footnotesize, color=Orange!35]{\textbf{\color{Orange}{\thetodocounter: Hieu: }} #2}}
}

% TODO MIKKEL
\newcommand{\mikkel}[2][ ]
{
  \ifthenelse{\equal{#1}{inline}}
    {\stepcounter{todocounter}\todo[inline, size=\footnotesize, color=BlueGreen!35]{\textbf{\color{BlueGreen}{\thetodocounter: Mikkel:} } #2}}
    {\stepcounter{todocounter}\todo[size=\footnotesize, color=BlueGreen!35]{\textbf{\color{BlueGreen}{\thetodocounter: Mikkel:} } #2}}
}

% TODO JONAS
\newcommand{\jonas}[2][ ]
{
  \ifthenelse{\equal{#1}{inline}}
    {\stepcounter{todocounter}\todo[inline, size=\footnotesize, color=Red!35]{\textbf{\color{Red}{\thetodocounter: Jonas:} } #2}}
    {\stepcounter{todocounter}\todo[size=\footnotesize, color=Red!35]{\textbf{\color{Red}{\thetodocounter: Jonas:} } #2}}
}

% TODO MARK
\newcommand{\marka}[2][ ]
{
  \ifthenelse{\equal{#1}{inline}}
    {\stepcounter{todocounter}\todo[inline, size=\footnotesize, color=CarnationPink!35]{\textbf{\color{CarnationPink}{\thetodocounter: Mark:} } #2}}
    {\stepcounter{todocounter}\todo[size=\footnotesize, color=CarnationPink!35]{\textbf{\color{CarnationPink}{\thetodocounter: Mark:} } #2}}
}

% Custom kommando til at ændre Bibliography således den vises som "part" i ToC
\renewcommand{\bibsection}{%
\chapter*{\bibname} % Her kan man ændre overskrift-typen til Bibliography - * angiver at den ikke skal vises i ToC
\bibmark
\ifnobibintoc\else
\phantomsection % Dummy-sektion så hyperref ved hvor bibliography-links skal henvise til
\addcontentsline{toc}{part}{\bibname} % Tilføj til ToC
\fi
\prebibhook}

% Til at lave en tekstboks
\definecolor{prettyboxgray}{RGB}{238,238,238}
\newenvironment{prettybox}{%
  \def\FrameCommand{\fboxsep=\FrameSep \fcolorbox{white}{prettyboxgray}}%
  \color{black}\MakeFramed {\FrameRestore}}%
{\endMakeFramed}

\definecolor{red}{HTML}{F4CCCC}
\definecolor{blue}{HTML}{C9DAF8}
\definecolor{green}{HTML}{D9EAD3}
\definecolor{grey}{HTML}{F3F3F3}

% Inkluder makroer
% !TEX root = rapport.tex

\newcommand{\secref}[1]{\ref{#1} \nameref{#1} p. \pageref{#1}}
\newcommand{\coderef}[1]{Code \ref{#1}}
\newcommand{\doccoderef}[1]{\ref{#1}}
\newcommand{\lineref}[2]{\ref{#1}, l. #2}

% Margin
\setlrmarginsandblock{3.5cm}{2.5cm}{*}
% \setlrmarginsandblock{Indbinding}{Kant}{Ratio}
\setulmarginsandblock{2.5cm}{3.0cm}{*} % \setulmarginsandblock{Top}{Bund}{Ratio}
\checkandfixthelayout % Laver forskellige beregninger og sætter de almindelige længder op til brug ikke memoir pakker

% Sørger for LaTeX ikke strækker teksten ved at forhindre der bliver tilføjet linieskift, hvis siden ikke er fyldt helt ud.
\raggedbottom

\addbibresource{Literaturereview.bib}
\addbibresource{Paper.bib}