% !TEX root = ../literature.tex
\section{Introduction}
Weiser\cite{Weiser:1991} points out:
 
{\em``Ubiquitous computing names the third wave in computing, just now beginning. First were mainframes, each shared by lots of people. Now we are in the personal computing era, person and machine staring uneasily at each other across the desktop. Next comes ubiquitous computing, or the age of calm technology, when technology recedes into the background of our lives.''}.

Weiser's vision of ubiquitous computing, is over 20 years old, includes the ubiquity availability of computers that are preferably not distinguishable from everyday objects.He also talks about "calm technology", this technology which resides in the periphery and plays a non-dominant role in a user's life. The perception of devices plays a major role in Weiser's vision, however the other side of that is availability of devices and also computing power, which is important for cross-device interaction. As so we could contend and regard that the latest incarnation of Weiser's vision is cross-device interaction, where ideally joining several devices would lead to single seamless, and Natural User Interface (NUI), flexible, and not restricted to a few configurations\cite{Radle:2015}.
In order for this interaction to happen we need to remain as close as possible to the real world and have multimodality in mind, as Jain et al. states: ``human interaction with the world is multi-modal, and rich multi-modal interaction is part of what defines a natural experience.''\cite{Jain:2011}. 
%You need to bring in the nui angle more subtly, after you quote from 12 - than you talk about natural and give the Wigor and Wixon definition of nui and how that responds to that quote
This corresponds well witth the way Wigdor and Wixon defines NUI , where "Natural" is about how the users feel and what they do when using a product. The products must mirror the user's capabilities and meet their needs, but the trick is to help the users feel comfortable about the product without the need for much practice.\cite{Wigdor:2011} This becomes important for the design of product that has a non-dominant role in a user's life as the user should not be using a lot of time for adapting to the products.   
Ideally we would see a combination of multiple modalities, for instance gestures, augmented reality, touch, voice recognition etc. 
Researchers have made and published breakthroughs and designers constructed innovation design in each modality individually, however there has only been little research and work in combining them.\cite{Jain:2011} \\

It is possible for the existence of untapped opportunities in the integration of multiple natural user interaction modalities for enriching the cross-device experience with hand-held devices and large public displays. \\

In this paper we review the research that has been done within interaction with large displays. 
This has been done to map the current understanding and practices, thereby helping future researchers, within this field, to gain a perspective as well as identifying possible untapped opportunities for future work.