% !TEX root = ../paper.tex
\section{Introduction} \label{sec:introduction}
The evolution of ways people interact with the digital world is noteworthy considering the short life-span of computing. How we use our devices, which devices we use, and the context in which we use them has been continually under transformation. 
From portable personal computers originally considered mostly for specialized field applications such as accountancy, military use, or for sales representatives, which addressed mobility of a person's workspace, to modern hand-held devices which presents their users with such degrees of freedom that ultimately workspaces are becoming more ubiquitous. 
In non-work context people are now connected mostly everywhere, which aids us in a search for information or in communicating, changing the way we interact. 
This expansion has not only increased mobile computing due to greater convenience, but also made it widespread.\cite{Francis:1997} 

As numerous divergent devices are being adopted in different domains and contexts, understanding cross-device interaction is currently becoming more important and relevant; after all, people take their hand-held devices into situations where other technologies are active. This ubiquitous presence of devices means that they can be used to enhance everyday situations in all kind of places. Imagine if a public display could morph from a one-way broadcast device that merely shows visual content to a two-way interaction device that provides a more engaging and immersive experience. This emergence of cross-device communication opportunities prompts a need to understand how different interaction techniques perform in use, i.e. in terms of how easily, quickly and accurately, or in terms of how enjoyable or satisfying it is to interact in this way.

Research in the area of cross-device interaction is increasing with the changing trends. Earliest examples are in the late 90's, within ubiquitous computing, with Rekimoto's work.  He argued for what he called multi computer user interface and that interaction techniques must overcome the boundaries among devices in multi-device settings\cite{Rekimoto:1998}.

Recent HCI research has focused on how to include natural modality more in cross-device interaction, contributing to what should be know as cross-device natural user interaction.  Some researchers used spatial information \cite{Marquardt:2011, Marquardt:2012}, others used touch \cite{Seifert:2012}, or combined touch with air gestures \cite{Bragdon:2011}. But we still have limited understanding of how to design cross-device natural user interaction techniques and we lack empirical studies of this.

Inspired by the opportunities presented by such challenges, this paper reports on a empirical study between four different cross-device natural user interaction techniques for data transfer between mobile devices and large displays.
We discovered that out of the four techniques we developed and implemented, \swipe was the most effective technique. We also show that even though \pinch is not as effective as the others, users described it as a fun technique.
