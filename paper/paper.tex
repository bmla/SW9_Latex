\documentclass{sigchi}

% Use this command to override the default ACM copyright statement
% (e.g. for preprints).  Consult the conference website for the
% camera-ready copyright statement.

%% EXAMPLE BEGIN -- HOW TO OVERRIDE THE DEFAULT COPYRIGHT STRIP -- (July 22, 2013 - Paul Baumann)
% \toappear{Permission to make digital or hard copies of all or part of this work for personal or classroom use is      granted without fee provided that copies are not made or distributed for profit or commercial advantage and that copies bear this notice and the full citation on the first page. Copyrights for components of this work owned by others than ACM must be honored. Abstracting with credit is permitted. To copy otherwise, or republish, to post on servers or to redistribute to lists, requires prior specific permission and/or a fee. Request permissions from permissions@acm.org. \\
% {\emph{CHI'14}}, April 26--May 1, 2014, Toronto, Canada. \\
% Copyright \copyright~2014 ACM ISBN/14/04...\$15.00. \\
% DOI string from ACM form confirmation}
%% EXAMPLE END -- HOW TO OVERRIDE THE DEFAULT COPYRIGHT STRIP -- (July 22, 2013 - Paul Baumann)

% Arabic page numbers for submission.  Remove this line to eliminate
% page numbers for the camera ready copy
% \pagenumbering{arabic}

% Load basic packages
\usepackage{balance}  % to better equalize the last page
\usepackage{graphics} % for EPS, load graphicx instead 
\usepackage[T1]{fontenc}
\usepackage{txfonts}
\usepackage{mathptmx}
\usepackage[pdftex]{hyperref}
\usepackage{color}
\usepackage{booktabs}
\usepackage{textcomp}
% Some optional stuff you might like/need.
\usepackage{microtype} % Improved Tracking and Kerning
% \usepackage[all]{hypcap}  % Fixes bug in hyperref caption linking
\usepackage{ccicons}  % Cite your images correctly!
% \usepackage[utf8]{inputenc} % for a UTF8 editor only

%OUR OWN REF PACKAGES
\usepackage{hyperref} %doublecheck the need for this one, this tex file has a global style for URL
\usepackage{cleveref}


% If you want to use todo notes, marginpars etc. during creation of your draft document, you
% have to enable the "chi_draft" option for the document class. To do this, change the very first
% line to: "\documentclass[chi_draft]{sigchi}". You can then place todo notes by using the "\todo{...}"
% command. Make sure to disable the draft option again before submitting your final document.
\usepackage{todonotes}

% Paper metadata (use plain text, for PDF inclusion and later
% re-using, if desired).  Use \emtpyauthor when submitting for review
% so you remain anonymous.
\def\plaintitle{Insert long clever title here}
\def\plainauthor{First Author, Second Author, Third Author,
  Fourth Author, Fifth Author, Sixth Author}
\def\emptyauthor{}
\def\plainkeywords{Authors' choice; of terms; separated; by
  semicolons; include commas, within terms only; required.}
\def\plaingeneralterms{Documentation, Standardization}

% llt: Define a global style for URLs, rather that the default one
\makeatletter
\def\url@leostyle{%
  \@ifundefined{selectfont}{
    \def\UrlFont{\sf}
  }{
    \def\UrlFont{\small\bf\ttfamily}
  }}
\makeatother
\urlstyle{leo}

% To make various LaTeX processors do the right thing with page size.
\def\pprw{8.5in}
\def\pprh{11in}
\special{papersize=\pprw,\pprh}
\setlength{\paperwidth}{\pprw}
\setlength{\paperheight}{\pprh}
\setlength{\pdfpagewidth}{\pprw}
\setlength{\pdfpageheight}{\pprh}

% Make sure hyperref comes last of your loaded packages, to give it a
% fighting chance of not being over-written, since its job is to
% redefine many LaTeX commands.
\definecolor{linkColor}{RGB}{6,125,233}
\hypersetup{%
  pdftitle={\plaintitle},
% Use \plainauthor for final version.
%  pdfauthor={\plainauthor},
  pdfauthor={\emptyauthor},
  pdfkeywords={\plainkeywords},
  bookmarksnumbered,
  pdfstartview={FitH},
  colorlinks,
  citecolor=black,
  filecolor=black,
  linkcolor=black,
  urlcolor=linkColor,
  breaklinks=true,
  hypertexnames=false
}

% create a shortcut to typeset table headings
% \newcommand\tabhead[1]{\small\textbf{#1}}

% End of preamble. Here it comes the document.
\begin{document}

\title{\plaintitle}

\numberofauthors{3}
\author{%
  \alignauthor{Leave Authors Anonymous\\
    \affaddr{for Submission}\\
    \affaddr{City, Country}\\
    \email{e-mail address}}\\
  \alignauthor{Leave Authors Anonymous\\
    \affaddr{for Submission}\\
    \affaddr{City, Country}\\
    \email{e-mail address}}\\
  \alignauthor{Leave Authors Anonymous\\
    \affaddr{for Submission}\\
    \affaddr{City, Country}\\
    \email{e-mail address}}\\
}

\maketitle

% !TEX root = ../paper.tex
\begin{abstract}
\todo[inline]{Write abstract}

\todo{Add section numbers to each section because of references to certain sections of the paper}

\end{abstract}


\category{H.5.m.}{Information Interfaces and Presentation
  (e.g. HCI)}{Miscellaneous} \category{See
  \url{http://acm.org/about/class/1998/} for the full list of ACM
  classifiers. This section is required.}{}{}

\keywords{\plainkeywords}

% !TEX root = ../paper.tex
\section{Introduction} \label{sec:introduction}
The evolution of ways people interact with the digital is noteworthy considering the short life-span of computing. How we use our devices, which devices we use, and the context in which we use them has been continually under transformation. From portable personal computers, originally considered mostly for specialized field applications, such accountancy, military use, or for sales representatives, which addressed mobility of a person's workspace, to modern hand-held devices which presented, their users, such degree of freedom that ultimately workspaces are starting to fade. This expansion has not only increased mobile computing due to greater convenience, but also made it widespread.\cite{Francis:1997} \\

As numerous divergent devices are being adopted in different domains and context, cross-device interaction is currently becoming more important and relevant, as people take their hand-held devices into situations where other technologies are active. This ubiquitous presence of devices means that it can be leverage to enhance everyday situations in all kind of places. Imagine if a public display can morph from one-way broadcast device that merely show visual content to a two-way interaction device that provides more engaging and immersive experience. Given this emergence of cross-device communication opportunities prompts a need to understand how different interaction techniques perform in use, e.g. in terms of how easily, quickly and accurately, or in terms of how enjoyable or satisfying it's it to interact in this way. \\

Research in the area of cross-device interaction has been trying to keep up with the changing trends, earliest examples are in the late 90's, within ubiquitous computing, with Rekimoto's work,  he argued for, what he called, multi computed user interface and that interaction technique must overcome the boundaries among devices in multi-device settings\cite{Rekimoto:1998}.

Recent HCI research has focused on how to include in cross-device interaction an more natural modality contributing to what should be know as cross-device natural user interaction.  Some researchers used spatial information \cite{Marquardt:2011, Marquardt:2012}, others used touch \cite{Seifert:2012}, or combined touch with air gestures \cite{Bragdon:2011} . But we still have limited understanding of how to design cross-device natural user interaction techniques and also we are missing on quantitative studies of the aforementioned.\\

Inspired by the opportunities presented of such challenges, this paper reports on a quantitative study between four different cross-device natural user interaction techniques for data transfer between hand-held device and large displays.

We discover that out of the four techniques we developed and implemented, \swipe was the most precise and effective technique, which could be used to create more interesting and effective natural user interfaces with. We uncover that the need for precision comes at a great cost in regards to effectiveness, and should be side-stepped\todo{contrast this with any other that was strongest at some aspect?? also, be clear about what you mean by effective.} if possible while designing and implementing the application.  



%\subsection{References and Citations}
%
%Use a numbered list of references at the end of the article, ordered
%alphabetically by last name of first author, and referenced by numbers
%in
%brackets~\cite{acm_categories,ethics,Klemmer:2002:WSC:503376.503378}.
%Your references should be published materials accessible to the
%public. Internal technical reports may be cited only if they are
%easily accessible (i.e., you provide the address for obtaining the
%report within your citation) and may be obtained by any reader for a
%nominal fee. Proprietary information may not be cited. Private
%communications should be acknowledged in the main text, not referenced
%(e.g., ``[Borriello, personal communication]'').
%
%References should be in ACM citation format:
%\url{http://acm.org/publications/submissions/latex_style}. This
%includes citations to internet
%resources~\cite{acm_categories,cavender:writing,CHINOSAUR:venue,psy:gangnam}
%according to ACM format, although it is often appropriate to include
%URLs directly in the text, as above.


% Use a numbered list of references at the end of the article, ordered
% alphabetically by first author, and referenced by numbers in
% brackets~\cite{ethics, Klemmer:2002:WSC:503376.503378,
%   Mather:2000:MUT, Zellweger:2001:FAO:504216.504224}. For papers from
% conference proceedings, include the title of the paper and an
% abbreviated name of the conference (e.g., for Interact 2003
% proceedings, use \textit{Proc. Interact 2003}). Do not include the
% location of the conference or the exact date; do include the page
% numbers if available. See the examples of citations at the end of this
% document. Within this template file, use the \texttt{References} style
% for the text of your citation.

% Your references should be published materials accessible to the
% public.  Internal technical reports may be cited only if they are
% easily accessible (i.e., you provide the address for obtaining the
% report within your citation) and may be obtained by any reader for a
% nominal fee.  Proprietary information may not be cited. Private
% communications should be acknowledged in the main text, not referenced
% (e.g., ``[Robertson, personal communication]'').

% Balancing columns in a ref list is a bit of a pain because you
% either use a hack like flushend or balance, or manually insert
% a column break.  http://www.tex.ac.uk/cgi-bin/texfaq2html?label=balance
% multicols doesn't work because we're already in two-column mode,
% and flushend isn't awesome, so I choose balance.  See this
% for more info: http://cs.brown.edu/system/software/latex/doc/balance.pdf
%
% Note that in a perfect world balance wants to be in the first
% column of the last page.
%
% If balance doesn't work for you, you can remove that and
% hard-code a column break into the bbl file right before you
% submit:
%
% http://stackoverflow.com/questions/2149854/how-to-manually-equalize-columns-
% in-an-ieee-paper-if-using-bibtex
%
% Or, just remove \balance and give up on balancing the last page.
%
\balance{}

% REFERENCES FORMAT
% References must be the same font size as other body text.
\bibliographystyle{SIGCHI-Reference-Format}
\bibliography{paper}

\end{document}

%%% Local Variables:
%%% mode: latex
%%% TeX-master: t
%%% End:
