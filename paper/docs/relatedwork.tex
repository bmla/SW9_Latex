% !TEX root = ../paper.tex
\section{Related work} \label{sec:relatedwork}

As digital signage is an inherently visual medium, graphics and animations are increasingly used to visualize data from business applications or other feeds. In fulfilling its capability the need of change from non-interactive information broadcasters to active mediums needs to happen.  Combining this with the recent trend of increased number of hand-held device per capita leads to new opportunities for exploring cross-device interaction (CDI) and hence design better interaction techniques. As such possibility are presented researchers naturally acted on them.\\

Perhaps one of the earliest working systems illustrating CDI is by Myers et al. \cite{Myers:2001} one of the application realized within their \emph{Pebbles project} is SlideShow Commanded that utilized Personal Devices Assistants (PDA) to control a PowerPoint presentation runny on other computer or laptop. It was possible not just moving between slides, but also scribbling and writing on the PDA slides, while annotations are shown on the presentation for the audience.\\


Hamilton et al. created a framework for cross device interaction in \emph{Conductor} . \cite{Hamilton:2014} While their main focus here was creating cross device interaction between multiple interactive devices from the same type - tablets. Their study expanded our knowledge of how people interact with multiple devices when presented with the possibility of cross device interaction. \emph{JuxtaPinch} by Nielson et al. \cite{Nielsen:2014} investigate the the use of multiple device from different type together, by allowing a number of devices being put next to each other and "pinched" together to form a larger collaborative workspace. Expanding on the idea of collaborative workspace was moving from using multiple devices to build one large mixed device, to having more than one or more input devices towards one or more output device. Schmidt et al. propose a cross-device interaction style for mobiles and surfaces. The researchers point out that \emph{"natural forms of interaction have evolved for personal devices that we carry with us (mobiles) as well as for shared interactive displays around us (surfaces) but interaction across the two remains cumbersome in practice"} \cite{Schmidt:2012}; so in order to overcome this they propose to use mobiles as tangible input on the surface in a stylus like fashion. Skov et al. \cite{Skov:2015} illustrate six different cross-device interaction techniques for the case of card playing, where the player sees the cards they can play on their phone and moves them to the common tablet that serves as playing table. They continue by quantitatively evaluating each of the techniques.  \\


As large displays become cheaper and more frequent people research interest increased.In \emph{Shoot and Copy}, Boring et al. \cite{Boring:2007} explored the transfer of data from a large public display onto a mobile device. They created a method of transferring data from a large screen by using the camera on the mobile device. The user would take a picture of whatever content they were interested in, and would query the content server with the picture they took. Through visual analyses, the content server would determine what content the user was interested in and would return that content to them. They show that there is a need for enabling data exchange between mobile devices and public displays. In \emph{Scroll, Tilt or Move It}, Boring et al. \cite{Boring:2009} investigated cross device interaction between large displays and mobile phones. More specifically, he investigated three different interaction techniques in order to continuously control a pointer on a large screen from a mobile device. \emph{Move} and \emph{Tilt}, two of the three interaction techniques, enabled faster selection time compared to the last one, \emph{Scroll}, but at the cost of higher error rates. They showed that different interaction techniques have certain strengths and weaknesses, and depending on the context and use, certain techniques can be used rather than other. 

Data transfer between devices is an area of interest in cross-device interaction. Marquardt et al. \cite{Marquardt:2012} study cross-device interaction on tablets with a extra layer of naturalness, by involving spatial information - proxemics.  Based on the constructs of f-formation and micro-mobility and co-present collaboration, they build their prototype with the idea of support for fluid and minimally disruptive interaction in document transfer.Bragdon et al.\cite{Bragdon:2011} propose using a combination of air gestures and touch. They aim to design, implement and test a system that allows a group of users to interact using air gestures and touch gestures. 
The purpose is to increase control, support democratic access, and share items across multiple personal devices such as smartphones and laptops where the primary design goal is fluid, democratic sharing of content on a common display.