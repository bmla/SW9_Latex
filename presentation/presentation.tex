\documentclass[10pt]{beamer}
\usetheme[
%%% options passed to the outer theme
%    hidetitle,           % hide the (short) title in the sidebar
    hideauthor,          % hide the (short) author in the sidebar
%    hideinstitute,       % hide the (short) institute in the bottom of the sidebar
%    shownavsym,          % show the navigation symbols
%    width=2cm,           % width of the sidebar (default is 2 cm)
%    hideothersubsections,% hide all subsections but the subsections in the current section
%    hideallsubsections,  % hide all subsections
%    left                % right of left position of sidebar (default is right)
  ]{Aalborg}
  
% If you want to change the colors of the various elements in the theme, edit and uncomment the following lines
% Change the bar and sidebar colors:
%\setbeamercolor{Aalborg}{fg=red!20,bg=red}
%\setbeamercolor{sidebar}{bg=red!20}
% Change the color of the structural elements:
%\setbeamercolor{structure}{fg=red}
% Change the frame title text color:
%\setbeamercolor{frametitle}{fg=blue}
% Change the normal text color background:
%\setbeamercolor{normal text}{bg=gray!10}
% ... and you can of course change a lot more - see the beamer user manual.

\usepackage[utf8]{inputenc}
\usepackage[english]{babel}
\usepackage[T1]{fontenc}
% Or whatever. Note that the encoding and the font should match. If T1
% does not look nice, try deleting the line with the fontenc.
\usepackage{helvet}
\usepackage{array} % needed for \arraybackslash
\usepackage{graphicx}
\usepackage{adjustbox} % for \adjincludegraphics


% colored hyperlinks
\newcommand{\chref}[2]{%
  \href{#1}{{\usebeamercolor[bg]{Aalborg}#2}}%
}

\title[Cross Device Interaction between Mobile Devices and Large Displays]% optional, use only with long paper titles
{Cross Device Interaction between Mobile Devices and Large Displays}

\subtitle{}  % could also be a conference name

\date{\today}

\author[]{%
  Bjarke M. Lauridsen  Ivan S. Penchev  Elias Ringhauge  Eric V. Ruder
}
% - Give the names in the same order as they appear in the paper.
% - Use the \inst{?} command only if the authors have different
%   affiliation. See the beamer manual for an example

%\institute[
%%  {\includegraphics[scale=0.2]{aau_segl}}\\ %insert a company, department or university logo
%  Dept.\ of Electronic Systems\\
%  Aalborg University\\
%  Denmark
%] % optional - is placed in the bottom of the sidebar on every slide
%{% is placed on the bottom of the title page
%  Department of Electronic Systems\\
%  Aalborg University\\
%  Denmark
%  
%  %there must be an empty line above this line - otherwise some unwanted space is added between the university and the country (I do not know why;( )
%}

% specify the logo in the top right/left of the slide
\pgfdeclareimage[height=1cm]{mainlogo}{AAUgraphics/aau_logo_new} % placed in the upper left/right corner
\logo{\pgfuseimage{mainlogo}}

% specify a logo on the titlepage (you can specify additional logos an include them in 
% institute command below
\pgfdeclareimage[height=1.5cm]{titlepagelogo}{AAUgraphics/aau_logo_new} % placed on the title page
%\pgfdeclareimage[height=1.5cm]{titlepagelogo2}{AAUgraphics/aau_logo_new} % placed on the title page
\titlegraphic{% is placed on the bottom of the title page
  \pgfuseimage{titlepagelogo}
%  \hspace{1cm}\pgfuseimage{titlepagelogo2}
}

\makeatletter
\def\blfootnote{\gdef\@thefnmark{}\@footnotetext}
\makeatother

\begin{document}
% the titlepage
{\aauwavesbg
\begin{frame}[plain,noframenumbering] % the plain option removes the sidebar and header from the title page
  \titlepage
\end{frame}}
%%%%%%%%%%%%%%%%

% TOC
\begin{frame}{Table of Content}{}
\tableofcontents
\end{frame}
%%%%%%%%%%%%%%%%

\section{Overview}
\subsection{Introduction}
\begin{frame}{Overview}{Introduction}

\end{frame}

\subsection{The idea of Synchronous Gestures}
\begin{frame}{Overview}{Synchronous Gestures}

\end{frame}

\begin{frame}{Overview}{Synchronous Gestures}

\end{frame}

%\begin{frame}{Overview}{Synchronous Gestures}
%\includegraphics[width=0.5\textwidth]{images/bumpGraphs.jpg} 
%\includegraphics[width=0.5\textwidth]{images/otherGraphs.jpg}
%\end{frame}

\begin{frame}{Overview}{Synchronous Gestures}

\end{frame}

\begin{frame}{Overview}{Synchronous Gestures}

\end{frame}


\section{Results presented in the paper}
\begin{frame}{Results presented in the paper}{}

\end{frame}

\section{Related work}
\subsection{Pick-and-Drop}
\begin{frame}{Related Work}{Pick-and-Drop (UIST 1997)}

\end{frame}

\subsection{ConnecTables}
\begin{frame}{Related Work}{ConnecTables (UIST 2001)}

\end{frame}

\subsection{The Triangles System}
\begin{frame}{Related Work}{The Triangles System (CHI 1998)}

\end{frame}

\section{In relation to research}
\begin{frame}{Importance in relation to the research (1)}{Synchronous Gestures in research}

\end{frame}

\begin{frame}{Importance in relation to the research (2)}{Synchronous Gestures in research}

\end{frame}

\begin{frame}{Importance in relation to the research (3)}{Synchronous Gestures in research}

\end{frame}

%\item Conductor: Enabling and Understanding Cross-Device Interaction (CHI 2014)
%		\begin{itemize}
%			\item Conductor is a prototype framework which serves as an exemplar for the construction of cross-device applications
%			\item Interaction methods by which users can easily share information, chain tasks across devices, and manage sessions across devices.
%		\end{itemize}

\section{Semester project and synchrony}
\begin{frame}{Synchronous Gestures in relation to our semester project}{}

\end{frame}

\section{Relevance and use in practice}
\begin{frame}{Use in practice}{}

\end{frame}

%%%%%%%%%%%%%%%%

{\aauwavesbg%
\begin{frame}[plain,noframenumbering]%
  \finalpage{Thank you! \\ \vspace{0.2cm} }
\end{frame}}
%%%%%%%%%%%%%%%%

\end{document}
