\section*{Introduction}
\addcontentsline{toc}{section}{Introduction}

Our motivation for this semester's project began with us being interested in Natural User Interfaces(NUI) in public spaces, with regards to the interaction between the users and the system, as well as with each other.
For example, a project that sparked our interest was one that was being conducted at Aalborg University. 
They were creating a large touch wall where multiple users could come up and create music together by touching the wall in different places. \\\\
This lead us to our initial project idea: a collaborative NUI painting application using tangible objects as painting tools. This idea was too broad though and was based too heavily on a field we did not have that much expertise in; namely creative arts.\\\\
We then decided to focus on interaction between people and large public displays. This lead us to our current project theme: researching cross-device interaction between mobile phones and large displays. \\\\
Since this is a relatively new research field, this gives us the opportunity to provide foundational research for use with large shared public displays in applications such as bulletin boards and interactive advertising boards.\\\\
The first paper is a literature review over the current state of the art in regards cross-device natural user interaction for mobile devices and large displays. In this work we analytically summarized relevant research outlining the current field and potentially identifying knowledge gaps. We look at works related to large displays, public displays, cross-device interactions, natural user interactions and the implication of public spaces on user interaction. We then use the findings to solidify our understanding of this knowledge area and set the foundation for our research paper.  \\\\
The second paper presents a research experiment regarding four different cross-device natural user interaction techniques. These four techniques were being used in research papers and their prototype applications. Here we explore these techniques based on their success rate, efficiency, accuracy and ease of use. We do this by creating an experiment in which users have to hit several targets with each technique. We then present our result and discuss their implications.