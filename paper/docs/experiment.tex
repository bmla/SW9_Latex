% !TEX root = ../paper.tex
\section{Experiment}
The four cross device interaction techniques mentioned above where implemented and then evaluated in a lab study in order to judge their performance compared to each other.

\subsection{Participants}
In total, 53 people took part in our experiment, which was conducted at Aalborg University's usability lab. The participants where between 20-45 years old (M: 24.4, SD: 4.3) and were between 1.63 and 1.95 meters tall (M: 182, SD:7.8). 88.7\% of users where right handed, and 96.2\% of them where smart phone users. Users had owned smart phones for 2-15 years (M:5, SD:2.1). The participants had several different backgrounds, but a very large part of them where university students. This was not formally recorded. They were recruited through a mixture of our social network, recruitment posters around the campus, and going through group rooms and asking for participants. 

\subsection{Experimental Design}
The experiment was conducted as a withing-subject research, with the four different interaction techniques and two grid sizes as independent variables. For the dependent variables, different measures of completion time and accuracy were used, as well as a small questionnaire in regards to the given interaction technique. Which technique started the test was randomized in order to mitigate the learning effect on the entire set of tests. In the end, the \pinch gesture started 25.4\% of all tests, \swipe started 22.6\% of all tests, \throw started 24.5\% of all tests, and \tilt started 26.4\% of all tests.

These where the following measures that were logged throughout the experiment: 

\textit{Practice Time:} This was the time each user spent during the practice portion of the experiment for each interaction technique. This was measured as the time from where user started the test for a given interaction technique until the user had hit his 3\ts{rd} target. 

\textit{Total Time:} This was the time each user spent completing the test for a given interaction technique.  This was measured from the time each user had hit his first target after the practice period until he had hit his last target. There were a total of 18 targets, plus the first target. 

\textit{Time per target:} This was the time each user spent hitting each of the target. This was simply measured as the time since each user last hit a target until he hit the current one.

\textit{Accuracy:} Whether or not each user hit the given target. Current pointer and target position (in pixels) were also recorded in order to give a more precise image of accuracy for each test. 

\textit{USE Questionnaire:} Each user was given a questionnaire after having gone through each interaction technique. There were 6 questions, all taken from the USE questionnaire \todo{Create proper reference}. These were asked to get an understanding of how useful and easy to use each technique was. The 6 questions were the following: 

%Arnold M. Lund, Measuring Usability with the USE Questionnaire, STC Usability SIG Newsletter, orginally published in the October 2001 issue (Vol 8, No. 2)
%http://garyperlman.com/quest/quest.cgi?form=USE 

\begin{itemize}
	\item It is easy to use
	\item Using it is effortless
	\item It is easy to learn to use
	\item I can use it successfully every time
	\item I quickly became skillful with it
	\item I learned how to use it quickly
\end{itemize}

User were able to rate their answers to each question on a 7 level Likert scale. We also wrote down any comments we heard during the experiment in order to get a qualitative understanding of each technique.  

\todo[inline]{Write about the the experiment here; Design, implementation etc.}