% !TEX root = ../literature.tex
\subsection{Cross-Device Interaction Natural User Interface}
Infusing the two fields could offer new concepts and potential ideas, as such an inquiry of the current literature would be beneficial. 
Marquardt wants {\em``to create techniques that allow people to seamlessly and naturally connect to and interact with the increasing number of digital devices.''} \cite{Marquardt:2011}. 
In order to achieve that, it is proposed to use spatial information - proxemics {\em``[...] to mediate people's interactions with digital devices, such as large digital surfaces or portable personal devices.''} \cite{Marquardt:2011}.

Marquardt et. al. later expand this idea by increasing the number of dimensions, he then creates: \\

{\em``[...]a system that explores cross-device interaction using two sociological constructs. First, F-formations concern the distance and relative body orientation among multiple users, which indicate when and how people position themselves as a group. Second, micro-mobility describes how people orient and tilt devices towards one another to promote fine-grained sharing during co-present collaboration.''} \cite{Marquardt:2012}.\\

Proxemics is one of the ways to create cross-device natural user interaction.
Seifert et al. propose the use of touch, they state that {\em``[...] it is envisioned that the tables in our domestic environments will turn into interactive surfaces once the price per square meter is in the region of a few hundred Euro.''} \cite{Seifert:2012}. 
Therefore, having touch capabilities on both input and output device will support co-located cross-device collaboration.
Bragdon et al. propose using a combination of air gestures and touch. 
They aim to design, implement and test a system that allows a group of users to interact using air gestures and touch gestures. 
The purpose is to increase control, support democratic access, and share items across multiple personal devices such as smartphones and laptops where the {\em``primary design goal is fluid, democratic sharing of content on a common display.''} \cite{Bragton:2011}.